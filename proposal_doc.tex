

\documentclass[english]{report}
%book
%article
%report
\usepackage[margin=1.25in]{geometry}
\usepackage[T1]{fontenc}
\usepackage{textcomp}
\usepackage[utf8]{inputenc}
\usepackage{babel}
\usepackage{float}
\usepackage{calc}
\usepackage{graphicx}
\usepackage{setspace}
%\usepackage[unicode=true]
% {hyperref}

\usepackage[hidelinks]{hyperref}
%\usepackage{hyperref}

\usepackage{url}
\usepackage{amssymb,mathtools}
%\usepackage{listings}
%\usepackage{multicol}
%\setlength{\columnsep}{1cm}
\usepackage{courier}
\usepackage{changepage}
%\usepackage{table}
\usepackage{acro}
%\usepackage[nottoc]{tocbibind}
\usepackage{listings}
\usepackage{relsize}
%\usepackage{etoolbox}
%\makeatletter
%\patchcmd{\chapter}{\if@openright\cleardoublepage\else\clearpage\fi}{}{}{}
%\makeatother


\acsetup{first-style=short}

\DeclareAcronym{RNN}{
	short = RNN ,
	long  = Recurrent Neural Network ,
	class = abbrev
}
%\lstset{basicstyle=\ttfamily,breaklines=false}

%\lstset{framextopmargin=50pt,frame=bottomline}

\makeatletter

%%%%%%%%%%%%%%%%%%%%%%%%%%%%%% LyX specific LaTeX commands.
%% Because html converters don't know tabularnewline
\providecommand{\tabularnewline}{\\}
%\patchcmd{\chapter}{\if@openright\cleardoublepage\else\clearpage\fi}{}{}{}
\makeatother

\doublespacing

\begin{document}

\pagenumbering{gobble}

\title{A Generative Chatbot with NLP}

\author{\noindent David Liebman david.c.liebman@gmail.com}



\date{\parbox{\linewidth}{\centering%
		\today\endgraf\bigskip
		Coordinator 1 \hspace*{3cm} Coordinator 2\endgraf\medskip
		Department of Computer Science \endgraf
		SUNY New Paltz}}

%Professor Name Name, Department of Computer Science, SUNY-New Paltz name@newpaltz.edu

\maketitle
%\pagebreak{}



\begin{center}
	


\section*{Abstract}
\end{center}

%\begin{adjustwidth}{1cm}{1cm}
We are interested in making a chat-bot. We want a computer program that can answer questions that might come up in a simple conversation.

We experiment with the Transformer Neural Network model and we try to explain in this paper how 
one works.

We chronicle a few experiments in Natural Language Processing. We try a GRU based chatbot. We 
try a transformer based chatbot. We also try a GPT2 based chatbot. 

We are also interested in installing the chatbot code on a small computer like a Raspberry Pi with speech recognition and speech-to-text software. In this way we might be able to create a device which speaks and which you can speak to. For the GRU based model we can use a Raspberry Pi 3B. In the case of GPT2 the running chatbot model uses too much ram, so we may try to install it on a Raspberry Pi 4B. GPT2 does not fit on a Raspberry Pi 3B.


%\end{adjustwidth}

\vspace{5mm}
%\section{Title of Project}
%Two Experiments for a Neural Network Chatbot


\newpage

\pagenumbering{roman}
\tableofcontents

\newpage
%\setlength{\linewidth}{350}
\listoffigures
\listoftables
%\begin{multicols}{2}
%\twocolumn[text]
%\twocolumn
\newpage
\pagenumbering{arabic}

	


\chapter{Background/History of the Study}

\chapter{Background/History of the Study}
\section{Background}
Transformer style models and also a Recurrent Neural Network model are used to allow for meaningful comparison. There is also the Generative Pre-Training 2 Transformer. \ac{RNN} models are explained in Chapter \ref{chapter-recurrent} and Transformers in are explained in Chapter \ref{chapter-transformer}.

It is worth noting that with the appearance of the Transformer architecture some traditional technologies have become redundant or obsolete. This may be true of any model that uses Recurrent Neural Network components and also the traditional word vector embeddings.

\subsection{Recurrent Neural Network and Transformer}

In their paper Vinyals et al \cite{DBLP:journals/corr/VinyalsL15} discuss making a chatbot using a neural network configured for Sequence-to-Sequence Neural Machine Translation. An attempt to code our own Sequence-to-Sequence model was not very fruitful so instead the thesis uses code authored by Inkawhich et al \cite{2018Inkawhich}.

In their paper Vaswani et al \cite{Vaswani2017AttentionIA} discuss using the Transformer architecture for solving machine learning tasks. A transformer model is trained as a chatbot.

In both of these experiments a common factor is the Movie Dialog Corpus that the models train on. The corpus comes from Danescu-Niculescu-Mizil et al \cite{Danescu-Niculescu-Mizil+Lee:11a}.

\subsection{Pre Trained Language Models}
Radford et al \cite{radford2019language} discuss the ``Generative Pre-Training 2'' (GPT2) neural network for Natural Language Processing (\ac{NLP}) tasks. The GPT2 model is based largely on the Transformer architecture. 

This is essentially a Language Model. The model is given a large section of text during training and it is asked to generate a single word or token to add to the end of the sample. During inference it does this over and over to complete a passage. This kind of model shows up in GPT2 and the thesis uses that trained model and some traditional programming techniques to generate text that naturally answers questions. Among all the models tested, this is the model that was found to be most comprehensive in execution.

Several chatbots are implemented, one with a GPT2 model using a program library from Wolf et al \cite{Wolf2019HuggingFacesTS} to run the model.

\section{Outline}
This thesis starts with explanations of how some Recurrent Neural Network components work. This is followed by a discussion of Transformers. Then the thesis discusses the GPT2 transformer.

The thesis goes on to describe installation of chatbot models on Raspberry Pi 4B computers, and on a Jetson Nano computer.

After that the thesis describes some further installations on an X86\_64 computer. Some graphs are given that show word and sentence usage of the Gated Recurrent Unit and Transformer based models. Some final thoughts are offered. Lastly some further reading is suggested for those interested.


\section{Goals For This Thesis}
This work focuses on providing a number of implementations of a Generative Chatbot, installed on small headless computers. Some overall goals are listed below. Checks in the check-boxes indicates the goal was achieved.

\begin{itemize}
	
	\item[\rlap{\raisebox{0.3ex}{\hspace{0.4ex}\tiny \ding{52}}}$\square$] Implement a generative chatbot using Recurrent Network components. This implementation uses GRU objects and the Tutorial found at Inkawhich et al \cite{2018Inkawhich}.
	
	\item[\rlap{\raisebox{0.3ex}{\hspace{0.4ex}\tiny \ding{52}}}$\square$] Implement a generative chatbot using a Transformer architecture and the Movie Dialog Corpus. This is discussed in Section \ref{transformer-movie-corpus}. %This implementation uses Scaled Dot Product attention.
	
	\item[\rlap{\raisebox{0.3ex}{\hspace{0.4ex}\tiny \ding{52}}}$\square$] Implement a chatbot using the Generative Pre-Trained 2 transformer. This is a pre-trained model. This is discussed in Section \ref{install-gpt2-chatbot}.
	
	\item[\rlap{\raisebox{0.3ex}{\hspace{0.4ex}\tiny \ding{52}}}$\square$] Subjectively compare the GRU, Transformer, and GPT2 models and the kind of output that each one produces. This is discussed throughout Section \ref{experiments-installations}.
	
	\item[\rlap{\raisebox{0.3ex}{\hspace{0.4ex}\tiny \ding{52}}}$\square$] Successfully install Google Speech Recognition and Text To Speech on the Raspberry Pi computer platform, as well as other small computers like the NVIDIA Jetson Nano. This is discussed in Section \ref{speech-and-speech-to-text}.
	
	\item[\rlap{\raisebox{0.3ex}{\hspace{0.4ex}\tiny \ding{52}}}$\square$] Install a GRU, Transformer, and GPT2 model on individual Raspberry Pi computers. Also install Google Speech Recognition and Speech To Text on these computers so that they can operate without keyboard or monitor. Allow the chatbot model to interact with a human's voice. This is discussed throughout Section \ref{experiments-installations}.
	
	\item[\rlap{\raisebox{0.3ex}{\hspace{0.4ex}\tiny \ding{52}}}$\square$] Install a GPT2 model on a Jetson Nano, along with Google Speech libraries and allow a human to interact with the chatbot. Compare execution time to the same model on the Raspberry Pi. This is discussed in Section \ref{chapter-nano}.
	
	\item[\rlap{\raisebox{0.3ex}{\hspace{0.4ex}\tiny \ding{52}}}$\square$] Compare the GRU and the Transformer model. Try to compare the word usage and sentence usage of the GRU and Transformer. This is discussed in Section \ref{gru-vs-transformer}.
	
	\item[\rlap{\raisebox{0.3ex}{\hspace{0.4ex}\scriptsize \ding{56}}}$\square$] Implement a generative chatbot using Recurrent Network components and an independent code base. This was not accomplished as the task was beyond the scope of this work.
	
	\item[\rlap{\raisebox{0.3ex}{\hspace{0.4ex}\scriptsize \ding{56}}}$\square$] Implement a generative chatbot using a Transformer architecture and the Persona chatbot corpus. This implementation did not work well. This is discussed in Section \ref{transformer-persona-corpus}. %uses Scaled Dot Product attention.
	
	
\end{itemize}



\chapter{GPT2 and Transformers}


\section{Transformer and Attention}

\label{transformer-intro}

The Transformer is a mechanism that is based entirely on attention. Strictly speaking this is not the attention explored in the Sequence to Sequence model in Section \ref{section-gru-attention}, though there are some similarities. It is a model that uses no recurrent components.

Recurrent components have some negative qualities. They are hard to run with batch input data. In addition they do not work with very long data strings. 

If a batch of data is to be cycled through a recurrent component, we would like all that batch to go through all the components at once.  We want this to happen for whole batches at a time.

The problem is that in Neural Machine Translation there are times when data has to be processed by the recurrent units one batch at a time. First the data is passed to the first RNN. Then the batch can be passed to the next RNN. Remember that a single GRU is used in our example for all words in the encoder. Another GRU is used in the decoder. These GRU's can not process the subsequent batches all at once. They have to wait for the previous batch to be done. This bottleneck has to do with how the RNN is used. 

We don't know how long each input sentence or output sentence is going to be. If sentences cannot be made to look like the same length to the RNN's, they must be handled individually. This ruins the batch concept.

These sorts of bottlenecks are found in Neural Machine Translation based on recurrent elements.

Because the Recurrent Network is so heavy with Neural Network components, many weights and biases, though they can remember patterns, they loose some information with every pass. This is why there is a practical limit to the length of the input sequences that the typical Recurrent Neural Network can use. This is why the length of sentences in network models that use the Recurrent Neural Network are short.

Transformers use no Recurrent Neural Network components. Their operations can be parallelized so that large batches of data can be processed at once during the same time step. 

Longer sequences can be considered as well, so Transformer input can contain longer English language sentences and even paragraphs. 

We will discuss one layer of a multi-layer Transformer below. Transformers are usually constructed on eight or more of these layers in the decoder and the encoder.

\subsection{Byte Pair Encoding}

\ac{BPE} stands for `Byte Pair Encoding.' WordPiece is a particular implementation of Byte Pair Encoding.

WordPiece is used by some Transformer systems to encode words much the way that Word2Vec does. Like Word2Vec, WordPiece  has a vocabulary list and a table of embeddings that maps one word or token to a vector of a given size.

WordPiece, though, handles Out Of Vocabulary (\ac{OOV}) words gracefully. It breaks large words into smaller pieces that are in the vocabulary, and has a special notation so that these parts can easily be recombined in order to create the input word again. Byte Pair Encoding is not so interested in pre-trained word embeddings like Word2Vec and Glove.

For the Generative Pre-training Transformer 2 a version of Byte Pair Encoding is used instead of a vocabulary system like Word2Vec or Glove.

Some form of Byte Pair Encoding is included in almost every Transformer typed Neural Network model, so no decision needs to be made about what type of word embeddings to use.

\subsection{Attention}
Attention mechanisms are used in a similar way in three places in the Transformer model. The first implementation of Self Attention is discussed below. Each of these attention mechanisms is contained in a layer. There are typically the same number of layers in the encoder as in the decoder.

It should be noted that input to the Transformer is strings of words from a desired input language. Output is similarly words in a given language. Input words are treated very much the way that they are in Sequence to Sequence models. A word is translated to a number and that number indexes an entry in a word-to-vector table. From that time forward a word is represented by a vector of floating point numbers. In a Transformer this word vector can be large. In the original paper Vaswani et al \cite{Vaswani2017AttentionIA} use a vector size of 512. Later, in our discussion of Generative Pre-Training 2, we will see vector sizes of 768 and 1280.

\begin{figure}[H]
	\begin{center}
		
		
		\includegraphics[scale=1.5]{diagram-mat04}
	\end{center}
	\caption[Transformer Encoder and Decoder]{Transformer Encoder and Decoder. - $Nx$ shows number of layers. - Vaswani et al \cite{Vaswani2017AttentionIA}}
	
	
\end{figure}

\subsection{Encoder - Scaled Dot-Product Attention}

Each layer of the Transformer's encoder has a signature self-attention mechanism. This is possibly one third of the entire Transformer mechanism, but a variety shows up in the other two-thirds. 

The first thing that happens is the input word vectors are converted to three other values. These new vectors are like the input vector but they have a smaller dimensionality. Converting the word vectors in this way is accomplished by three simple matrix multiplication operations.

In the diagram below a simple conversion of this type is illustrated. In the diagram we convert a vector with dimension of 1x3 to a dimension of 1x2. In a real world example we are converting a vector from 1x512 to 1x64. This is a division of 8.



\begin{figure}[H]
	\begin{center}
		
	
	\includegraphics[scale=0.5]{diagram-mat01}
\end{center}
	\caption[Lowering Dimensionality]{Lowering Dimensionality}
	

\end{figure}

In the Transformer, this conversion operation is probably the reason for the model's name. The input is \textit{transformed} to a lower dimension. 

One thing we want to do is to preserve the dimension of our starting vector. We start with a 512 sized floating point vector and after some processing we want to return to the same size. Before that is done the vector is processed at the smaller size of 64 floating point numbers. 

In this self-attention scheme three vectors are actually required. All three vectors are sized 64, and all three are converted by separate matrix multiplication operations. The weights to convert each of the three vectors are different. For this reason the new smaller vectors are all different.

The smaller vectors individually are called q, k, and v. They can also be referred to as larger matrices. The new vector matrices are denoted as Q, K, and V. Q stands for `Query'. K stands for `Key'. V stands for `Value'. The lower-case names refer to single vectors and the upper-case refer to matrices. These are essentially batches of input.

The Query value is multiplied by the Key values from all vectors in the input. This multiplication is `dot-product' multiplication. When it is done, all keys will have low output values, except those that are closest to the Query. Then the results are passed through a softmax function. When this is complete, there will be a single vector that is close to 1 and another group of vectors that are all close to 0.

The vector produced by multiplying the softmax with the V values of every word produces a single word vector that is close to its original value, and many others that are near zero.

This formula from Vaswani et al \cite{Vaswani2017AttentionIA} shows the process.

$$
\mathlarger{ \mathlarger{
Attention(Q,K,V)=softmax(\dfrac{QK^T}{\sqrt{d_k}})V
} }
$$

Here the value of $\sqrt{d_k}$ is used to limit the size of the $QK^T$ output and $d_k$ is the dimension 512. Without this the softmax function has to deal with much larger numbers. Smaller numbers for the softmax are preferred. $K^T$ is notation for the $K$ vector transposed.

The function can actually perform this on large matrices with high dimensionality, in parallel. This parallel matrix operation increases the speed of training.

In the triangle in the Figure \ref{attantion-7} we perform the multiplication and selection that was just described.

\begin{figure}[H]
	\begin{center}
		
		
		\includegraphics[scale=0.5]{diagram-mat07}
	\end{center}
	\caption[Attention Output]{Attention Output}
	
	\label{attantion-7}
\end{figure}




Finally the output calculated above must be returned somehow to the input dimensionality. This is accomplished by duplicating the procedure described eight times with eight separate weights. When this is done the output of the eight separate attention mechanisms is concatenated together, returning the output to the proper size.

This multi-headed approach allows different heads to learn different types of relationships, and then when they are grouped together the learned relations are recovered and contribute to the output.

\begin{figure}[H]
	\begin{center}
		
	
	\includegraphics[scale=0.5]{diagram-mat02}
\end{center}
	\caption[Matching Input and Output]{Matching Input and Output}
	\label{attention-matching}

\end{figure}


Later the output is passed through a feed forward network. It is also re-combined with the original input again through addition. Then the output is normalized. This makes sure that the values are all within reasonable ranges. This recombination of the attention output with the original output is done throughout each Transformer layer.

This describes the encoder section. There are two other attention segments. Together these two sections combine to form the decoder section. This is repeated for each layer.

\begin{figure}[H]
	\begin{center}
		
		
		\includegraphics[scale=1.0]{diagram-flow1}
	\end{center}
	\caption[Transformer Encoder and Decoder Flow]{Transformer Encoder and Decoder Flow. - Three layers and data flow.}
	\label{diagram-flow1}
	
\end{figure}

In our flow diagram, Figure \ref{diagram-flow1}, we are not concerned with most sub segments of the entire transformer. We are interested in the flow of data through the three encoder segments and the different flow of data through the decoder segments. The encoder is largely serial, while the decoder is serial and parallel. In fact each decoder segment includes a feed forward part, and all decoder and encoder parts include a residual connection where the input is added back to the output of the attention and feed forward segments.

The output of the encoder is a group of vectors the same size as the input sequence. They become the `Key' and `Value' batches below.

\subsection{Decoder Attention I - `Key' and `Value'}
The decoder is composed of two attention mechanisms and a feed-forward segment at each layer. The result of the encoder's work is passed to the decoder and remains applied to one of the decoder's attention mechanisms in each decoder layer. In one attention mechanism of the decoder the `Key' and `Value' matrices are imported from the encoder. 

While the encoder takes in the entire input and attends to whatever portion of that input it finds to be important, the decoder is interested in producing one output token at a time. 

In our flow diagram we illustrate one layer of the decoder.

\begin{figure}[H]
	\begin{center}
		
		
		\includegraphics[scale=1.25]{diagram-flow-decoder02}
	\end{center}
	\caption[Decoder Flow]{Decoder Flow Details - `I' for encoder-decoder attention. `II' for masked decoder self-attention.}
	
	
\end{figure}


We illustrated in the Sequence-to-sequence discussion the importance of the single `thought vector'. The Transformer can be seen as having a thought-vector also. There is a corridor of data from encoder to decoder. It is composed of a sequence of vectors the size of the input sequence or sentence. It is larger, strictly speaking, than a single vector.

Two important smaller vector-sized inputs from the encoder are ultimately required in all layers of the decoder. They represent the `Key' and `Value' matrices from the thought vector. The matrices required are the size of the smaller, reduced, vector. The full sized vectors are transported from the encoder and are reduced dimensionally in the decoder layers to a sequence of two smaller matrices. 

These full sized vectors come from the last encoder layer's output. Typically there will be as many decoder layers as there are encoder layers. The output from the last encoder layer is applied to the `Key' and `Value' inputs of one of the attention mechanisms in all the decoder layers.

\subsection{Decoder Attention II - `Query'}
There is another attention mechanism in each decoder layer. It works solely on data from the decoder itself. It works very much like the attention mechanism from the encoder - only it attends to every word of output as opposed to the entire input sequence. It passes its output to the attention mechanism described above. This data is lowered in dimensionality and becomes the `Query' matrix for that mechanism. 

The `Key' and `Value' sequences from the Encoder are a group of vectors the size of the input sequence. The `Query' matrix is the size of a single vector. This is because the Decoder is interested in predicting one word at a time. This section produces a single vector as well.

\subsection{Decoder Attention II - Masking}
Input for the second decoder attention section is a group of vectors from the output generated so far. During inference this output grows by one token with every pass through the decoder. This is how text is generated.

During training the second decoder section is masked. The mask prohibits the decoder from seeing parts of the target. This mimics the inference setup. In inference the decoder can only see up to the most recent word it has produced.

During inference the decoder produces a token and then it adds to that token, one at a time, until the decoding is finished and something like English is produced. It can attend to any part of the output it has already produced. It is concerned with producing a single token at a time.

These tokens strung together are the output of the Transformer. This output should be readable English if the model output is set up for the English language.

\begin{figure}[H]
	\begin{center}
		
		
		\includegraphics[scale=1.25]{diagram-mask01}
	\end{center}
	\caption[Decoder Mask]{Mask. - Decoder uses masked input during training.}
	
	\label{diagram-mask-01}
\end{figure}



%\subsection{Transformer - General}
\subsection{Input - Positional Encoding}
The input of the Transformer encoder and decoder layers employ not only a word-vector table, but also a positional encoding scheme. The model adds to the input vector information that it can then use to learn the position of words in a sentence. 

Words that are early in the sentence have a certain appearance and words later on appear differently. The Encoder and Decoder use sine and cosine waves to impart this information onto the sentence sequence. 

\subsection{Output - Feed Forward Network}
At the output of the last layer of the decoder the output vectors are processed through a linear matrix which increases the vector's dimensionality so that the output vector is the size of the output vocabulary dimensionality. After the linear matrix the vector is processed by a softmax function. Then the highest floating point value in the new larger vector is the index of the chosen output word.


\subsection{Visualization - Transformer}

In order to visualize what is happening during inference we have colorful charts that we can look at. In this chart we are looking at how each word attends to all the other words in the input text.

\begin{figure}[H]
	\begin{center}
		\includegraphics[scale=2]{Figure_3}
		
		
	\end{center}
	\caption[Visualized Attention]{Visualized Attention -- `favorite' shows attention to some but not all words in the sentence.}
	
	
\end{figure}

It is significant that words, like `what' and `your', do not have strong attention to other words in the text. In a chart like this one they would show no colors on the left and light colored lines connecting the right to the left.

This diagram is from the Transformer with the larger hyper-parameter set that we describe in Chapter 3, trained on the movie dialog corpus.


\section{The Generative Pre-Training 2 Model}

`Generative Pre-Training 2' (\ac{GPT2}) is a large model. It is based on the Transformer from Vaswani et al \cite{Vaswani2017AttentionIA} but there are some major changes. The model uses the decoder portion of the Transformer without the encoder. There are some other changes to the output layers. Another big difference is that it is pre-trained and downloadable.

\subsection{Pre-Training}
Pre-Training is when the authors of a model train an instance and then make the model available to the public on-line. This is helpful for the average programmer interested in Neural Networks. Training an instance of the Transformer model can use up computation resources for days, and require hardware that is costly. Usually the cost of producing a trained model is prohibitively expensive.

After acquiring a trained model, the programmer goes on to adjust the model to their task. Adjusting a pre-trained model to a given task is called `Transfer Learning'. Many tasks lend themselves to Transfer Learning. Conceptually a model can be fine-tuned to any problem and many problems can be addressed with good results after only modest fine-tuning.


\subsection{General}
GPT2 still uses Scaled Dot-Product Attention. A model diagram is taken from Radford et al \cite{radford2018improving}. A mask is used in the Self Attention segment of the model during training.

\begin{figure}[H]
	\begin{center}
		
		
		\includegraphics[scale=3.0]{diagram-mat05}
	\end{center}
	\caption[Generative Pre-Training 2 ]{GPT2 - Radford et al \cite{radford2018improving}}
	

\end{figure}

There are several sizes of pre-trained GPT2 model. They are all rather large. The smallest model has 12 layers while the Transformer model in the example from Vaswani et al \cite{Vaswani2017AttentionIA} uses 8 layers. This model also has a hidden dimension of 768, not 512. With 8 heads this leaves a smaller dimensionality of 96 at each attention head. 

The GPT2 models input and output text sequences.


\subsection{Training}

The GPT2 model is trained on text from the web, specifically Reddit. The goal for training is to show the model part of a large piece of text and then to have the model predict the next word. 

For this task the mask is important. Training could consist of incrementally showing the model text at different states of completion and then asking it to predict the next token. In this kind of arrangement the batch sizes would be shorter and focus on each word. 

On the other hand training could present the model with the text in complete form and have the model look at it through the mask. A mask is visualized in Diagram \ref{diagram-mask-01}.

Each word would have an opportunity to be focused on as the `next' word. A boundary is formed between the last word and the masked area to its right. The boundary between each word and the one that follows it is explored and the model can still be trained in large batches in parallel. Words to the right of the last word and the particular boundary being looked at are not available to the model.

Training is done by the developers of the model and the authors of the paper. The model is too big for individuals to train from scratch. 

\subsection{Inference}

In this example we will discuss creating `conditional samples'. This is in contrast to creating `unconditional samples'. Conditional samples rely on an input sequence for generating output. Unconditional samples have no input specified. A mask is not used during inference.

First we must select for our example a series of input tokens. This series of tokens are generated from an English sentence of our choosing. The sequence we use will be `Good day' for this example. We will assume that the words in the sequence translate into single tokens in the corpus. It could be that an input word is made up of several tokens, but our example is simple enough that that should not be the case.

Assume that the input context for our GPT2 model is 768 tokens. This is reasonable to assume for the 117M model. Our input tokens, the words `good' and `day' take up two spots in the input area. They are followed by an end-of-sequence token. Together they all take up three spots. At that time there are 765 spots left in the input area.

The first words are converted to tokens. They are passed through the embedding matrix where they are converted to vectors. Positional encoding patterns are generated for each of the three vectors. These positional encoding patterns are created from sine and cosine waves that are concatenated together. They are added to the input tokens.

The model starts at the 4th location and attempts to generate the 4th token. The entire model is at this moment addressing the task of generating the 4th token.

One of the important processes that the input goes through is the Scaled Dot-Product Attention. This is performed at each layer. There are 12 layers in the 117M model.

All three tokens are converted to smaller vectors for each layer. Then the third vector, for the end-of-sequence token, is treated as the `Query'. The matrices for the `Key' and `Value' are assembled from the words in the input. This is done for all of the heads of each individual layer.

At this time the model is addressing the transition from the third spot to the fourth spot.

The third word `Query' vector, the end-of-sequence, is compared to each other previous word `Key' vector using dot-product multiplication. Then the result is Softmaxed. This produces a single vector that is close to 1 and a group of all other vectors that are closer to 0. The result of that is multiplied by all of the three `Value' vectors. A single result is found in this way.

The output is concatenated together at each layer across all the heads at that layer. Ultimately the output is recombined with the input. There are also components in each layer that do normalization.

Ultimately an output vector is produced that represents the model's best guess at what the next token should be. This vector is converted into a size equivalent to the size of the vocabulary. Then the model can use some method to choose a word from the vocabulary. 

Frequently, the model looks to the largest floating point number and its association with a word in the vocabulary.

The new token is placed in position four. The first three tokens are left as they are and GPT2 goes back to the start and looks at, now, the first four tokens as input. It will try to generate the fifth token.

The model will continue to try to generate tokens until the input area is filled and there are 768 tokens, or until there is a special `end-of-sequence' token generated. The output could be anywhere from 1 to 765 tokens long. This is because the input area starts with a dimension of 768, and there are three tokens in the original input sentence.


\begin{figure}[H]
	\begin{center}
		
		
		\includegraphics[scale=2.0]{diagram-inference-01}
	\end{center}
	\caption[Generative Pre-Training 2 Inference]{GPT2 - Inference Progress}
	
	
\end{figure}

\subsection{Corpus}
The GPT2 models are trained on a corpus called WebText. WebText is a 40GB corpus that is taken from the Reddit web site. All the material comes from before 2017 and all the material has a `karma' rating of 3 or better. `Karma' is a rating system used internally on Reddit. 

%As with the decoder layer of the Transformer model, the GPT2 model concerns itself with generating words that are later strung together to make sentences or paragraphs. During training the model uses a masking scheme so that input can be parallel-ized. During inference output cannot be parallel-ized, so during inference output must focus on one example at a time.

\subsection{Releases}
In their paper Radford et al \cite{radford2019language} show that their model can generate text from a seed sentence or paragraph. At the time the case was made that the largest `Generative Pre-Training 2' models should not be released because of their ability to generate text that might fool humans into believing that another person was responsible for the text. Later the larger models were released to the public.

\begin{center}

\begin{tabular}{lrll}
	Size & Parameters & Layers & $d_{model}$ \\
	\hline
	small & 117M       & 12     & 768          \\
	medium & 355M       & 24     & 1024         \\
	large & 774M       & 36     & 1280         \\
	x-large & 1.5B     & 48     & 1600 \\
	xx-large & 8.3B   &  72 &   3072 
\end{tabular}

	
\end{center}
\addcontentsline{lot}{section}{GPT2 Size Overview}

At the time that the first `Generative Pre-Training 2' model was released the size of the models was mis-stated, but the documentation was not updated immediately. Most values in the table above show sizes that were actually released. The final xx-large model was trained by NVIDIA Applied Deep Learning Research \cite{2019NVIDIAadlr} and was not released to the public.

The `Generative Pre-Training 2' models also work in many circumstances in `zero-shot' mode. This is when you use the pre-trained model but without transfer learning. There is no extra training that goes on to make the model suit the task. It is used `as is'.

For our chatbot the model with 117 million parameters worked. Some programming was required to make the model output look like chatbot output, but the model itself was not modified.

We use both the small and large model. As a test, when the larger 774M model was released it was used as a substitution for the 117M model. The test worked, and returned answers that were more well formed than the small model. The larger model does not fit on a Raspberry Pi and so it was not employed on a permanent basis. Using the extra large 1.5B parameter model in a chatbot was not attempted at first.

\subsection{Application Details}
The model is described in Radford et al \cite{radford2019language} and the accompanying blog post. The model is trained on English without a stated problem. Large neural network models are usually trained with a stated problem in mind. Rather famously this model is used after training to generate English language text. The model takes input from the user, a premise or summary of what is to be generated. The model also takes as input a number called the `temperature.' Then the model generates output. As the `temperature' is set higher the output is more fanciful. There is also a tune-able parameter for the output length. 

Given the ability of the model to invent content, it was determined by the authors that the 'large' model should not be released to the public at first. Months later the 'large' model was released. 

For our chatbot we set the temperature to a low number. We set the length of the output to a sentence-length number of tokens. Then as input we use the output from the speech-to-text translator.

The output is interesting but not useful right away. Traditional programming and string manipulation are employed to clean the output and render a short single sentence. This is our final output.

Because the input is meant to be a number of sentences, and because we are using a Transformer-based architecture, we have room in the input string to add more information along side the user's question. In this respect the model acts to summarize the input. 

With every input string we include a set of three or four sentences. They include the time, the bot's name, and the bot location and occupation. All of these are invented. What happens is the chatbot summarizes the input and only if the information is relevant then the same information is used by the model as output. Making this possible is the fact that a Transformer can accept much longer input strings than a Gated Recurrent Unit, and generate much longer output strings.

Surprisingly the chatbot answers most of our questions in the first person. We feel that WebText, the Reddit corpus, has many examples of sentences in the first person.

\subsection{Visualization - GPT2}

During inference the Scaled Dot Product Attention in the GPT2 focuses on certain words as it processes input text. Here the word `favorite' shows a relationship to many of the other words in the text.  

\begin{figure}[H]
	\begin{center}
		\includegraphics[scale=2]{Figure_4}
		
		
	\end{center}
	\caption[Visualized Attention GPT2]{Visualized Attention GPT2 -- `favorite' shows attention to some but not all words in the sentence.}
	\label{diagram-vis04}
	
\end{figure}

In our experiments the phrase `What is your favorite color?' is often answered with `I love the colors of the rainbow.' This answer does not mention a specific color, as one might expect it should. Figure \ref{diagram-vis04} might support this observation because `color' on the left is not heavily highlighted. Words like `what' and `your' are barely considered at this head at all. 



\chapter{Experiments}

\section{Approach to the Study}

Several tasks are necessary for the project. One task is to implement the algorithms for different models, one the sequence to sequence model and the other the transformer model for a generative chatbot and finally the Generative Pre-training Transformer 2 model.

We will not try to rewrite the transformer or Generative Pre-training Transformer 2 model ourselves.

In this project we attempt to load as much of out chatbot code onto a Raspberry Pi as possible. We have trained models using the pytorch and tensorflow libraries. These models are responsible for taking in English sentences and producing English output. There is another part of the typical Raspberry Pi setup that includes another neural network component. Speech to text models, which our application requires, rely on large neural network resources. For this purpose we use speech to text resources supplied by Google on the google cloud. To include speech to text libraries locally on the Raspberry Pi would be too costly in computation time and resources like RAM. It would also be complicated to implement technically. It could easily comprise an entire project on its own.

Unfortunately the speech to text resources supplied by Google cost money. To use the service you need to have a billing account with Google.

The speech to text service used on the project and the memory limitations on the Raspberry Pi leads one to ask the question weather the neural network responsible for the chatbot function could not be servable from some faster machine located somewhere on the internet. At this time we are not interested in serving these resources. It would entail two calls from the Raspberry Pi for every sentence. This complicates things and also has a time overhead. 

Also, we have several models that we want to test. To test them all would require several servers. Also we use both Pytorch and tensorflow. Tensorflow has `tensorflow-model-server' for serving models, but Pytorch has no equivalent.

It is important to note that the large Generative Pre-training Transformer 2 model specifically could be served from a remote computer and it would operate faster. Currently on the Raspberry Pi decoding a single sentence takes approximately 13 seconds. Even so, we prefer to install our trained models on the Raspberry Pi directly.

\section{Model Overview}

%\begin{center}

\begin{table}[h!]
	
	\begin{center}
		
		
		\begin{tabular}{llllll}
			
			Model Name    & File  & RAM  & RAM  & Pretrained  & Hand  \\
			&  Size & Train   & Interactive   & Weights & Coded   \\
			\hline
			\hline
			Seq-2-Seq/Tutorial & 230 M     & 1.1 G & 324 M & NO                 & NO        \\
			Transformer/Persona   & 25 M      & 556 M & 360 M & NO         & PARTIAL         \\
			Transformer/Movie   & 550 M      & 6.5G & 1.5G & NO         & PARTIAL         \\
			GPT2 small*   & 523 M     & 5 G   & 1.5 G & YES                & NO        \\
			\hline
		\end{tabular}
		
		* a large GPT2 model exists, but it is not small enough to fit on a raspberry pi.
		
		
	\end{center}
	
	\label{fig:modeloverview}
	\addcontentsline{lot}{section}{Model Overview}
\end{table}
%\end{center}

Here we itemize a short description for each row in the table.

\begin{itemize}
	\item \textbf{Sequence to Sequence - Tutorial} This model uses the sequence to sequence architecture and the Gated Recurrent Unit component. We hand-coded our own example of this model but it performed poorly. This model is the slightly modified version of the Sequence to Sequence model based on the tutorial from Inkawhich et al (2018)\cite{2018Inkawhich}. It actually uses the Movie Dialog corpus.
	\item \textbf{Transformer - Persona} This model uses a Tensorflow Transformer architecture. There was some coding involved to get the model to interface with the text-to-speech and speech-to-text libraries. There was also some coding to load our own corpus data during training. The model parameters describe a rather small model. This model also uses the Persona Dialog corpus.
	\item \textbf{Transformer - Movie} This model is based on the transformer model above but uses the Movie Dialog corpus and a parameter set that is larger. In many ways this model is bigger than the model that uses the Transformer and the Persona corpus.
	\item \textbf{GPT2 small} This model was downloaded from the internet. It fits on a Raspberry Pi 4B with the 4GB RAM option. Some modification was made so that model output was suitable for our purposes.
\end{itemize}



\section{Setup}

We use linux computers, sometimes with \ac{GPU} hardware for parallel processing. We also use the Python programming language. Code from this project can be run with the 3.x version of Python.

When the project was started we did some programming with Keras using
Tensorflow as a backend. Keras was later discarded in favor of Pytorch.
Pytorch as a library is still under development at the time of this
writing.

Some of the Generative Pre-training Transformer 2 code uses Pytorch. Some of the Transformer and Generative Pre-training Transformer 2 code uses Tensorflow. There is a repository on Github that has the GPT2 trained model using Pytorch instead of Tensorflow.

We use github as a code repository. Code corresponding with this paper can be found at: \href{https://github.com/radiodee1/awesome-chatbot}{https://github.com/radiodee1/awesome-chatbot}
. 

As a coding experiment we rewrite the code for the sequence-to-sequence Gated Recurrent Unit model. We have varying amounts of success with these experiments. We do not rewrite the Generative Pre-training Transformer 2 code from the Tensorflow or Pytorch repository.

\section{Graphical Processing Unit vs. Central Processing Unit}

A CPU has a number of cores, a number usually between 2 and 16. A CPU is designed, though, to execute one command at a time. This allows for a logical chain of actions that can be programmed for execution. A CPU has limitations when it comes to executing matrix multiplication. Matrix multiplication using a CPU can take a long time.

GPUs, Graphical Processing Units, have the ability to address tasks like matrix multiplication with many more processing units at once. The GPU speeds up parallel processing and have a benefit to neural networking training tasks that the CPU doesn't have.

Unfortunately state of the art neural network models are larger than the capacity of a single GPU. Some models are trained on many GPUs simultaneously. It is not uncommon for a model to train on a computer with eight GPU cards for many days. Training these models is prohibitively expensive for the average programmer. It is possible to rent time on Amazon Web Services or Google cloud with well outfitted computers but this can be costly.

This sort of situation is addressed partially by the Transfer Learning scheme. In Transfer Learning someone else trains the model and makes the trained version accessable to the public. Then the average programmer downloads the model and fine tunes it to their task.

This would be fine if there were a model for every task. Though many exist there seems to be many tasks that are not adressed. It seems that there is always the opportunity to train a larger model by utilizing the CPU and training for long periods of time. This arrangement is not advised for the sort of experimentation where it is not a certainty that the output will be successfull. If the goal is to do somehting that might not work, don't undertake it or use a Amazon or Google computer. If success is assured and time is plentiful continue with the CPU.

In this paper the GRU based Sequence-to-sequence model and the Tensorflow based Transformer model were trained from scratch on a CPU laptop. In the case of the Transformer, several days were required for training.

\section{Raspberry Pi}

A Raspberry Pi is a small single board computer with an `arm' processor. There are 
several versions on the market, the most recent of which sports built-in wifi and
on-board graphics and sound. The memory for a Raspberry Pi 3B computer is 1Gig of RAM. Recently
available, the Raspberry Pi 4B computer can sport 4Gig of RAM.

It has always been the intention that at some time some chatbot of
those examined will be seen as superior and will be installed and
operated on a Raspberry Pi computer. If more than one model is available
then possibly several models could be installed on Pi computers.

For this to work several resources need to be made available. Pytorch
needs to be compiled for the Pi. Speech Recognition (\ac{SR}) and Text
To Speech (TTS) need to work on the Pi.

For the transformer model to work Tensorflow needs to work on the Pi.

All the files that are trained in the chosen model need to be small
enough in terms of their file size to fit on the Pi. Also it must
be determined that the memory footprint of the running model is small
enough to run on the Pi.

In the github repository files and scripts for the Raspberry Pi are
to be found in the \textquoteleft bot\textquoteright{} folder.

Early tests using Google\textquoteright s SR and TTS services show
that the Pi can support that type of functionality. 

Google's SR service costs money to operate. Details
for setting up Google's SR and TTS functions is beyond
the scope of this document. Some info about setting this up can be
found in the README file of this project\textquoteright s github
repository.

The pytorch model that is chosen as best will be trained on the
desktop computer and then the saved weights and biases will be transferred
to the Raspberry Pi platform. The Pi will not need to do any training,
only inference. 



\section{Tensorflow vs. Pytorch}

Tensorflow is a Google library. Pytorch has it's roots with Facebook. Both run in a Python environment. The learning curve for Tensorflow is steeper than for Pytorch. Pytorch offers the programmer python objects that can be combined to create a neural network. Tensorflow has different pieces that can be combined, but they cannot be examined as easily at run time.

Tensorflow has a placeholder concept for inputting data and getting back results. You set up these placeholders at design time. They are the only way of accessing your data at run time.

Pytorch objects interact with Python more naturally. You can use print statements in your code to show data streaming from one object to another. This is possible at run time.

In favor of Tensorflow, it has a good tool for visualization which can print out all kinds of graphs of your data while your model trains. It is called Tensorboard.

\section{Speech and Speech To Text}

Google has python packages that translate text to speech and speech to text. In the case of text to speech the library is called `gTTS'. In the case of speech to text the library is called `google-cloud-speech'. 

The gTTS package is simple to use and can be run locally without connection to the internet. The google-cloud-speech package uses a google cloud server to take input from the microphone and return text. For this reason it requires an internet connection and an account with Google that enables Google cloud api use. Google charges the user a small amount for every word that they translate into text. 

Both of these resources, the text-to-speech and speech-to-text, work out of the box on the Raspberry Pi, but configuring speech-to-text for the Pi is not trivial. The user must register a billing account with Google cloud services. In return for this registration the user is allowed to download a json authentication file. The file must be copied to the Raspberry Pi. 

Furthermore an environment variable must be set that points to the authentication file. The variable is called `GOOGLE\_APPLICATION\_CREDENTIALS'. This environment variable has to be set up before the respective model runs. When the model is launched on startup it may not be launched as a regular user. The model may be launched as, for example, the root user. Somehow the environment variable must be set along with the launching of the neural network model.

The operating system on the Raspberry Pi is based on Debian Linux. In this operating system there is a file which is run immediately after the basic system starts up. This script is called `\textbf{/etc/rc.local}'. It is sufficient to put the environment variable there and follow it with the launching of the model. To ensure that the process goes without a hitch, we attempt to combine the setting of the environment variable with the launching of the program in a single line of code.

\section{ARMv7 Build/Compile}

\subsection*{Pytorch `torch' Library 1.1.0 For ARMv7}
We compile the Pytorch library for Raspberry Pi. We use several virtualization techniques to do this compilation. The result of those efforts is a Pytorch python 3.7 library for the Raspberry Pi.

On their web site Milosevic et al (2019)\cite{2018Milosevic} compile Pytorch 1.1.0 for the Raspberry Pi. We follow their instructions closely. We are able to build the package for the ARMv7 platform.

The instructions called for constructing a change-root environment where a Fedora Core 30 linux system was set up. Then the ARMv7 system was used in the change-root environment to compile the Pytorch library for the 1.1.0 version.

The production laptop used for development ran Ubuntu linux. For this reason a Virtualbox emulation was set up with Fedora Core 30 on it. Inside that emulator the change-root environment was set up. The library was compiled there successfully. 

There are two problems with the resulting built python package. Firstly there is an error in python when importing the torch library. The error reads `ImportError: No module named \_C'. 

After some research it is clear that the build process for ARMv7 creates some shared object files that are misnamed. A hackey fix is to find the misnamed files and make copies of them with a suitable name. The same outcome could be assured by making symbolic links to the misnamed files with proper names.

There are three files misnamed. They can be found at `\textbf{/usr/lib/python3.7/site-packages/torch/}'. They are named with the same convention. They all have the ending `.cpython-37m-arm7hf-linux-gnu.so'. We want to rename them with the much shorter `.so'. The files are then named `\textbf{\_C.so}', `\textbf{\_dl.so}', and `\textbf{\_thnn.so}'.

This takes care of the `ImportError'. The second problem is that the version of GLIBC in the change-root environment does not match the GLIBC library in the Raspberry Pi Raspbian distribution. This produces the following error: `\textbf{ImportError: /usr/lib/x86\_64-linux-gnu/libstdc++.so.6: version `GLIBCXX\_3.4.26' not found}'.

This is solved by rebuilding the package with Fedora Core 29 instead of 30. 
 
\subsection*{Pytorch `torch' Library 1.4.0 For ARMv7}
We recompile the Pytorch library for the Raspberry Pi. We use debian virtualization techniques for the compilation. Because ubuntu is a debian derivative it is not necessary to run the process in a Virtualbox container. 

In addition to this, the files created by the compilation are properly named. There is no need to go to the directory `\textbf{/usr/lib/python3.7/site-packages/torch/}' to change anything. 

The time spent compiling the software is approximately 5 hours. Time spent with the Virtualbox container was easily twice that. The time spent on the Raspberry Pi executing a single Generative Pre-training Transformer 2 question and answer remains about 13 seconds, so there was no gain in that respect.

There were several small hurdles to completing the compilation. Firstly the `debootstrap' command needed to be employed at the start. Debian Stretch was used as the host operating system. It was felt that if it was used that the GLIBC compatibility problem would not be faced. This turned out to be the case.

There are some dependencies that need to be installed on the `chroot' environment for Pytorch to compile. One of these is that is important is `libblas3.'

Then Python 3.7 needed to be built on Stretch. The Stretch program repositories use Python 2.7 and 3.5 . The Rapbian operating system on the Raspberry Pi 4B is based on Debian Buster and uses Python 3.7. After compiling Python 3.7 the Git program needed to be compiled from scratch. Git on Stretch has a issue that is fixed upstream, but we want to use Stretch because of the GLIBC issue. Instead of using the upstream fix, we compile Git ourselves.

It is conceivable that the GLIBC issue would not be important if the `chroot' environment used Debian Buster, since that is the basis for the current Raspbian operating system. The Stretch operating system solution works though.

Finally the Pytorch program needed to be built. We disable CUDA and distributed computing as neither exists on the Raspberry Pi.

\subsection*{Docker Container `tensorflow-model-server' For ARMv7}
The Google machine learning library for python uses a standalone program called `tensorflow-model-server' for serving all tensorflow models in a standard way. The program has not been officially compiled for ARMv7. There exists, though, a docker image that will run on ARMv7.

Docker can be run on the Raspberry Pi in the ARM environment. Below is a terminal excerpt that shows how to do this.

\begin{verbatim}
$ sudo apt-get update
$ sudo apt-get upgrade
$ curl -fsSL test.docker.com -o get-docker.sh 
$ sh get-docker.sh
$ sudo usermod -aG docker $USER
\end{verbatim}

After the last command you need to log out and then log in again to take advantage of the newly installed docker.

The original idea was to follow someone else's instructions and compile the Docker Container for the ARMv7. Then the executable would be removed from the container and used natively in the Raspberry Pi.

It was found that there existed a version of the Docker Container Daemon that ran on the Raspberry Pi. All that remained was to write a Docker Container script that interacted with the existing ARMv7 container. The author of the original container is Erik Maciejewski (2020)\cite{2020Maciejewski}.

`Tensorflow-model-server' is used on the localhost internet address, 127.0.0.1, with a port of 8500. tensorflow-model-server is meant for serving neural network resources on the internet, but with careful planning it works on the Raspberry Pi.

\section{Experiments - Installations}
In the section above we describe the workings of a transformer and the workings of Generative Pre-training Transformer 2. We propose they are similar. Here we distinguish between the two. For the experiments section they are totally separate.

We have several basic neural network models. One is the basic sequence to sequence model typically used for neural machine translation. We also have the transformer and the Generative Pre-training Transformer 2. We try to touch on each model type and we also distinguish between chatbot operation and smart-speaker operation. This gives us six sections.

The Generative Pre-training Transformer 2 code is versatile. We use it for the chatbot problem. The model is pretrained. We experiment with transfer learning and further training of the Generative Pre-training Transformer 2 model, but in our case it does not improve the model's performance. 

When we talk about the chatbot problem on Generative Pre-training Transformer 2 we are talking about a PyTorch version of the GPT2 code. 

We found that the chatbot with the hand coded sequence to sequence Neural Machine Translation did not work very well. There is a tensorflow transformer-based model that worked marginally well. We found that the Generative Pre-training Transformer 2 chatbot worked well enough with the `zero-shot' setup so that fine-tuning was not necessary. 

Fine tuning on the Generative Pre-training Transformer 2 problem actually had a negative effect. To fine-tune the model also involved training the model on Tensorflow and then translating the model to PyTorch when that was done. 

We use speech recognition and speech to text libraries to allow a user to give the chatbot model auditory input.

We also train the transformer to do the chatbot task. This works better than the the hand coded sequence to sequence model but not as well as the Generative Pre-training Transformer 2. 

For the sake of experimentation we use a sequence to sequence Gated Recurrent Unit tutorial and implement the chatbot. This model works well and allows us to compare that model with the Generative Pre-training Transformer 2 model.

Finally we describe installing the chatbot in small computing platforms, the Raspberry Pi 3B and 4B, in an attempt to create a smart speaker.

All of the three mentioned models work on a laptop computer. For the laptop we experimented with a setup where the model would launch applications for the user when the user asked the model to. These audio cues were interesting but were in no way the focus of our experiments.

It should be noted that at some point we would like to describe some of our subjective results with the different models. It is difficult to measure our results objectively because we are using code most closely associated with a translation task. 

If we were measuring progress with an actual translation task accuracy could be measured as a score that improves when the output matches the meaning of the input, albeit in a different language. We don't use an actual translation task, we use something close to one. The problem is that though the input and the output are in the same language, the input and output have different meanings. We do not dictate what the correct output would be for a given input. The output just has to make sense as a reply in the English language. We would actually prefer that the output not have the same meaning as the input. This makes it difficult to calculate an objective accuracy score. This is touched on in the paper from Vinyals et al(2015)\cite{DBLP:journals/corr/VinyalsL15}.

Loss can still be monitored during training. When the loss stops decreasing you know you should stop training as the model is probably overfitting.

\subsubsection*{Questions}
Below is a list of questions asked of all models.

\begin{verbatim}
Hello.
What is your name? 
What time is it?
What do you do?
What is your favorite color?
Do you like red?
Do you like blue?
What is your favorite candy?
Do you like ice cream?
Good bye.
\end{verbatim}

Subjectively the transformer model did not perform as well as the Generative Pre-training Transformer 2 model, but it did perform better than our hand coded Gated Recurrent Unit model. The model from the Gated Recurrent Unit tutorial performed better than the initial Transformer model but not as good as Generative Pre-training Transformer 2.


\subsection{Chatbot - Gated Recurrent Unit Model}
We have two models based on the sequnce to sequence architecture proposed by Vinyals et al (2015)\cite{DBLP:journals/corr/VinyalsL15}. One model was largely written by hand. This first model does not work very well. The second model was taken largely from an online tutorial by Matthew Inkawhich (2018)\cite{2018Inkawhich}.

We trained the first sequence to sequence model on a large english
corpus in an attempt to produce a chatbot. 

For this sequence to sequence model we want to use text found in a movie dialog corpus (Danescu-Niculescu-Mizil et al, 2011)\cite{Danescu-Niculescu-Mizil+Lee:11a}. 

In our experience with coding our own chatbot we find that the model learns a single English sentence that can be used for most replies. 

Subjectively the Gated Recurrent Unit chatbot is unusable.

For example our chatbot frequently replies to input with the phrase `I don't know'. This is the most common output from our model. In this respect our original chatbot responds poorly.

A large portion of the time the bot's response to a question is the same. This is interesting in that the chatbot has identified that this phrase is a suitable answer to many questions, but it is disappointing that there is not more variety to the output.  

Occasionally the sequence-to-sequence model will reply with the phrase `I'm sorry.' This happens very infrequently and it is not clear why the model chooses to reply this way. 

The second sequence to sequence model performed exactly as expected. It answered a number of subjective questions with original, authentic answers. The answers all made sense in the English language.

\subsubsection*{Questions}
This is the sample question list as it was answered by the second model.

%\begin{minipage}{2in}
\begin{verbatim}
> hello .
you re a good man .
> what is your name ?
smith .
> what time is it ?
i don t know .
> what do you do ?
i m a comedian .
> what is your favorite color ?
i don t know .
> do you like red ?
no .
> do you like blue ?
no .
> what is your favorite candy ?
it s a big one .
> do you like ice cream ?
no .
> good bye .
good bye .
\end{verbatim}
%\end{minipage}
%\bigskip

 
\noindent \textbf{Checklist:} 


\begin{itemize}

\item[\rlap{\raisebox{0.3ex}{\hspace{0.4ex}\scriptsize \ding{52}}}$\square$] All the responses are in plain English. There is no gibberish.

\item[\rlap{\raisebox{0.3ex}{\hspace{0.4ex}\scriptsize \ding{52}}}$\square$] There is a variety of answers. Not all answers are the same.

\item[$\square$] It is debatable weather or not the answers to the questions about `favorite color' and `favorite candy' are good. The model could have a set of easy answers that it can use for this kind of question. 

\item[\rlap{\raisebox{0.3ex}{\hspace{0.4ex}\scriptsize \ding{56}}}$\square$] `No' is a safe answer for many types of question as it is clearly English, it follows logically, and it is short and easy to remember. Another safe answer is `I don't know'. This model uses that answer at times.

\item[\rlap{\raisebox{0.3ex}{\hspace{0.4ex}\scriptsize \ding{52}}}$\square$] The model answers well to `Hello' and `Good bye'.
\end{itemize}

\subsection{Smart Speaker - Gated Recurrent Unit Model}

The Gated Recurrent Unit model was installed on a Raspberry Pi. This allowed us to test out speech-to-text and text-to-speech libraries. The Raspberry Pi model was 3B. The RAM requirements were less than 500MB and the trained model answered questions on the Raspberry Pi almost instantaneously.

For this experiment we compiled the Pytorch library for Raspberry Pi.

The Raspberry Pi was outfitted with a microphone and a speaker and nothing more. It was also configured so that the Pytorch sequence to sequence model ran automatically on startup.

The model requires access to the internet for the exchange that the speech to text software has to make with the Google servers. If there is no internet the model doesn't work.

As there was no monitor and it took some time for the model to launch, the program was coded to beep when the model was ready to accept input.

\subsection{Chatbot - Transformer Model with Persona Corpus}
Using the Persona corpus we trained a transformer model to use as a chatbot. This transformer was not pre-trained with any large corpus, so this example did not use transfer learning. The Persona corpus comes from Mazar{\'{e}} et al(2018)\cite{DBLP:journals/corr/abs-1809-01984}.

This model uses the tensorflow library, not Pytorch, and a transformer model that is somewhat small.

The memory footprint of the model while it was running was below 1 Gigabyte. It is conceivable that the model could be installed on a Raspberry Pi board but it requires a python package called `tensorflow-model-server' and this package would have to be built from source for the Raspberry Pi. 

Training of the model followed a certain pattern. First the model was trained on the persona corpus until a familiar pattern emerged. When the model began to answer all questions with the 
phrase "I don't know" training was stopped. 

At that time the corpus was modified to include no 
sentences that have the word "don't" in them. Training was started again until the output contained nothing but the phrase "I'm sorry." 

At that time the corpus was modified to include no sentences that have the word "sorry" in them. Training was started again and was continued for some period. Training was stopped. A further segment of training was not attempted. 

At this point, after looking at the change in loss, further training was not thought of as helpful. Loss stopped improving at some point in this process, and this lack of improvement was taken as a sign that progress was not likely.

Subjectively the transformer model is better than the original Gated Recurrent Unit model. It is not necessarily better than the Gated Recurrent Unit model from the Sequence to sequence tutorial. It can respond to something like four sentences. When it comes upon a question that it doesn't expect it defaults to a certain sentence. It can answer questions that you might ask in a rudimentary conversation. It has answers to prompts like `hi', `How are you?' and `What do you do?'. If you tell it your name it will tell you that its name is `Sarah'. It doesn't answer arbitrary questions. It cannot answer 'What is your favorite color?'. It can not tell you the time. The default reply sentence for unknown prompts is `Hi, how are you today?'


%\begin{lstlisting}[language=bash]
\subsubsection*{Questions}
This is the sample question list as it was answered by the model.

\begin{verbatim}
> hello
hi , how are you today ?
> what is your name?
hi , how are you today ?
> what time is it ?
I like to read a lot
> what do you do ?
i'm a student
> what is your favorite color ?
hi , how are you today ?
> do you like red ?
hi , how are you today ?
> do you like blue ?
hi , how are you today ?
> what is your favorite candy ?
hi , how are you today ?
> do you like ice cream ?
yes , i do 
> good bye
hi , how are you today ?
\end{verbatim}

\noindent \textbf{Checklist:} 



\begin{itemize}
	
	\item[\rlap{\raisebox{0.3ex}{\hspace{0.4ex}\scriptsize \ding{52}}}$\square$] All the responses are in plain English. There is no gibberish.
	
	\item[\rlap{\raisebox{0.3ex}{\hspace{0.4ex}\scriptsize \ding{52}}}$\square$] There is a variety of answers. Not all answers are the same.
	
	\item[\rlap{\raisebox{0.3ex}{\hspace{0.4ex}\scriptsize \ding{56}}}$\square$] Some of the answers are re-used and do not follow logically from the questions. The `favorite color' and `favorite candy' questions are nearly ignored. For those questions the model answers with `Hi, how are you today?'. This seems to be the model's default answer.
	
	\item[\rlap{\raisebox{0.3ex}{\hspace{0.4ex}\scriptsize \ding{52}}}$\square$] The model does not use `No' or `I don't know'.
	
	\item[\rlap{\raisebox{0.3ex}{\hspace{0.4ex}\scriptsize \ding{56}}}$\square$] The model does not have an answer for `Good bye'.
\end{itemize}


\subsection{Smart Speaker - Transformer Model with Persona Corpus}

The transformer model is installed on the Raspberry Pi. This model is more dynamic than the GRU hand-coded model. 

The transformer model takes about two minutes to boot on the Raspberry Pi. After that the time between responses is slow. The time between the first two or three responses is uncomfortably slow. After those first responses the time between answers gets to be more natural.

There is one special tone that the Raspberry Pi gives at the end of loading the model. This tone notifies the user that the model is loaded and ready to respond to questions.


\subsection{Chatbot - Transformer Model with Movie Corpus}
Using the Movie corpus we trained a transformer model to use as a chatbot. This transformer was not pre-trained with any large corpus, so this example did not use transfer learning. 

This model uses the tensorflow library, not Pytorch, and a transformer model that is larger than the other Transformer based model that uses the Persona corpus.

The memory footprint of the model while it was running was above 1.5 Gigabyte. The model could be installed on a Raspberry Pi 4B board but it requires a python package called `tensorflow-model-server' and this package would have to be built from source for the Raspberry Pi. 

The model took about seven days to train with a CPU based processor. The goal for training was 50,000 lines from the movie corpus. After training the loss graph was consulted and the installed version was culled from the saved checkpoint at the 45,000 line point.


\begin{figure}[H]
	\begin{center}
		\includegraphics[scale=3.5]{Figure_2}
		
		
	\end{center}
	\caption[Loss - Larger Transformer Model]{Loss - Orange is training loss and blue is evaluation loss.}
	
	%\addcontentsline{lof}{section}{Word Embeddings}
\end{figure}

Subjectively this transformer model is better than the Transformer model based on the smaller hyper-parameter set and the Persona Corpus.


%\begin{lstlisting}[language=bash]
\subsubsection*{Questions}
This is the sample question list as it was answered by the model.

\begin{verbatim}
> Hello.
hello 
> What is your name?
i don't know 
> What time is it?
i don't know 
> What do you do?
what do you mean ?
> What is your favorite color?
i don't know 
> Do you like red?
no 
> Do you like blue?
yeah 
> What is your favorite candy?
i don't know 
> Do you like ice cream?
yeah 
> Good bye.
bye 
\end{verbatim}

\noindent \textbf{Checklist:} 



\begin{itemize}
	
	\item[\rlap{\raisebox{0.3ex}{\hspace{0.4ex}\scriptsize \ding{52}}}$\square$] All the responses are in plain English. There is no gibberish.
	
	\item[\rlap{\raisebox{0.3ex}{\hspace{0.4ex}\scriptsize \ding{52}}}$\square$] There is a variety of answers. Not all answers are the same.
	
	\item[\rlap{\raisebox{0.3ex}{\hspace{0.4ex}\scriptsize \ding{56}}}$\square$] The `favorite color' and `favorite candy' questions are ignored.
	
	\item[\rlap{\raisebox{0.3ex}{\hspace{0.4ex}\scriptsize \ding{56}}}$\square$] The model does in fact use `No' or `I don't know'. It also likes to answer `yeah'.
	
	\item[\rlap{\raisebox{0.3ex}{\hspace{0.4ex}\scriptsize \ding{52}}}$\square$] The model does have an answer for `Good bye'.
\end{itemize}

\subsection{Smart Speaker - Transformer Model with Movie Corpus}

The transformer model is installed on the Raspberry Pi. It takes about five seconds to answer any question.

The transformer model takes about two minutes to boot on the Raspberry Pi. After that the time between responses is slow. The time between the first two or three responses is uncomfortably slow. After those first responses the time between answers gets to be more natural.

There is a special tone that the Raspberry Pi gives at the end of loading the model. This tone notifies the user that the model is loaded and ready to respond to questions. The model is also configured to beep intermittently during operation to signal that it is processing an input. This is helpful for a configuration where there is no monitor.


\subsection{Chatbot - Generative Pre-training Transformer 2 Model}
We used a pre-trained Generative Pre-training Transformer 2 model with the large english corpus to produce a chatbot and ascertain if this model works better than the sequence-to-sequence model. In our tests this worked well. The corpus is called `WebText'.

For our experiments Generative Pre-training Transformer 2 was used for the chatbot model in `zero-shot' mode. This means we did no special fine-tuning of the model in the application.

We did do some special coding for the input and output code in order to operate it as a chatbot. Input and output was limited to about 25 tokens. 

Input to the model was prepended with the character string "Q:" by our code. Output was observed to have the character string "A:" prepended to it. We assume therefore that the model was at some point exposed to the "Question/Answer" paradigm in written passages during its training. This was helpful.

Output from the model was usually larger in size than we needed. Also, output had the character of having some sensible utterance followed by some output that was only a partial sentence.

It was necessary to `scrape' the output. First the output was checked for the "A:" character string at the start. If it was there it was removed. Then the first complete sentence was used as output, while words and phrases after that were discarded.

\subsubsection{Context Experiment}
We decided that we would attempt to give the model some details that it could draw on during normal execution. We had two choices here. One choice was to train the model using fine-tuning and transfer learning to recognize certain questions and to supply answers. The other choice was to simply show the model the list of facts that we thought were important before every input sequence. This information would be summarized with each reply.

The second choice was more interesting. The text that the model was shown always included the name of the model (picked somewhat arbitrarily) along with information about the location of the model and the occupation. The time was also included.

This worked under most circumstances. Occasionally the model would report the wrong time when asked. This sort of activity is mysterious.

\subsubsection{History Experiment}
We tried concatenating all input and output and including it with each question. The hope was that the model would treat the `Q' and `A' as historical input and then it could refer to that input as it answered new questions. This didn't work, and the model was usually confused by the extra input.

\subsubsection{Artificial Intelligence Markup Language Experiment}
It was deemed helpful if the model could be given a question and instructed how to answer it. To this end AIML files were constructed and an AIML kernel was employed. The user's question was shown to the AIML kernel and then the model was shown the kernel's output (if there was one) along with the original question. The hope was that the output could be controlled by the AIML component. It didn't work. The AIML confused the model, and the model would not reliably choose to answer with the AIML text, as it might with the time of day.

\subsubsection{Overall}

Subjectively the model was the best of those tested. The model would answer questions about it's location, it's name, and the time, faithfully most of the time. Interestingly there where times when it did not do so. Some times it used alternative answers. For example, it would answer with the time but not the correct time. This was odd.

Under almost all circumstances the output was sensible English. There were few if any times where the model replied with gibberish. 

The subject matter of the prompts did not need to be the same as the simple introductory conversation of the transformer model. In fact any subject matter could be chosen and the model would answer. The model did not remember its own answers but it was consistent. Questions it answered include `What is your favorite color?' and `Do you like lollipops?'. 

\subsubsection*{Questions}
This is the sample question list as it was answered by the model. Note that the information mentioned in the answer about the time was accurate when the test was run.

\begin{verbatim}
> hello
Hello.
> what is your name ?
My name is Jane.
> what time is it ?
02:59 PM January 28, 2020.
> what do you do ?
I am a student.
> what is your favorite color ?
I love the color of the rainbow.
> do you like red ?
Yes.
> do you like blue ?
I do.
> what is your favorite candy ?
I love candy.
> do you like ice cream ?
I do. 
> good bye
Good bye.
\end{verbatim}

\noindent \textbf{Checklist:} 


\begin{itemize}
	
	\item[\rlap{\raisebox{0.3ex}{\hspace{0.4ex}\scriptsize \ding{52}}}$\square$] All the responses are in plain English. There is no gibberish.
	
	\item[\rlap{\raisebox{0.3ex}{\hspace{0.4ex}\scriptsize \ding{52}}}$\square$] There is a variety of answers. Not all answers are the same.
	
	\item[$\square$] It is still debatable weather or not the answers to the questions about `favorite color' and `favorite candy' are good. The model could have a set of answers that it can use for this kind of question. The model seems to know what candy is and to a lesser extent what a color is. Some of the time the answer includes a word from the question sentence that would lead you to believe that this model has fewer stock answers. The answers are good but not perfect.
	
	\item[\rlap{\raisebox{0.3ex}{\hspace{0.4ex}\scriptsize \ding{52}}}$\square$] The model does not use `I don't know' that often. 
	
	\item[\rlap{\raisebox{0.3ex}{\hspace{0.4ex}\scriptsize \ding{52}}}$\square$] The model does have an answer for `Good bye'.
\end{itemize}

The model will answer with it's name and you can tell it your name, but it is confused by this. It will on occasion tell you that it's name and your name are the same thing. This is in part because it cannot remember what it most recently said to you or what you most recently said to it. 

\subsection{Smart Speaker - Generative Pre-training Transformer 2 Model}
Tests showed that the Generative Pre-training Transformer 2 chatbot worked well. We wanted to continue and allow the chatbot to have more of the abilities of a smart speaker. We constructed a simple corpus that contained key phrases that we wanted the chatbot to recognize and act upon. We did some transfer learning with this new corpus.

We found that one of two things would happen. The chatbot would either learn the new phrases and forget all it's pre-training, or it would not learn the new phrases and it would retain all it's pre-training. For our examples there seemed to be no middle ground. Comparisons were made with all available models and a version without the transfer learning was settled on.

Code was added that uses Text To Speech and Speech To Text libraries. In this way the model could interact with a subject using auditory cues and commands.

We did some programming that allowed the model to launch programs when directed to by the user. In this way we have tried to move our project closer to the smart-speakers that are produced commercially. The programming did not rely on the neural-network aspects of the model. Instead the code used heuristics and simple word recognition. This code can be disabled when the model is run from the command line.

The Raspberry Pi model that the Generative Pre-training Transformer 2 was installed on was the 4B with 4GB of RAM. It is largely for this model that we cross compiled the Pytorch Python library for the ARMv7. The GPT2 model fit on the Raspberry Pi. While execution on the production laptop was instantaneous, execution on the Raspberry Pi took about 13 seconds for every response from the neural network.

The Raspberry Pi was outfitted with a microphone and a speaker but no mouse, monitor, or keyboard. The program was modified so that there was a tone every time the model was ready for input. Without such a tone it would be difficult to know when to speak and when to wait for a response. Aesthetically this arrangement is not perfect, but it allows the Generative Pre-training Transformer 2 model to be physically installed on the Raspberry Pi.

\section{Observation}
It is important here to compare the GRU chatbot with the larger Transformer based chatbot. Using our subjective qualifications we see that the GRU model answers with more variety than the transformer model. The important observation is that the hyper-parameter set for the Transformer model can be expanded and enlarged as needed before training. The GRU model cannot be trained successfully with an arbitrarily large hyper-parameter set. We can train a larger Transformer and obtain the benefit associated with this, namely better responses.

A single further observation is that the GRU model responds very quickly, while a transformer model may take more time relatively. This is not a problem for general applications, but for our purposes we cannot ignore the time spent by the transformer model when it is installed on a small computer like a Raspberry Pi. 

\section{Tests}

\subsection*{Turing Test}

The Turing Test concerns itself with the question of weather a computer is intelligent. Turing says that intelligence is too hard to describe, and that if the computer can convince you that it is intelligent then it is.

Weather this is right is beyond the scope of this paper. The people who trained the Generative Pre-training Transformer 2 were apprehensive about their model's ability to generate human speech. At first when they finished their model they decided not to release the largest version to the public for several months (Radford et al.)\cite{radford2019language}. Ultimately they did release their large model.

The creators of the model used it differently than our chatbot implementation. They generated paragraphs of text, and it was determined at first that the ability of the model to impersonate a human was too great. It was felt that the model could be used to spam facebook and other social networking sites with content that was very convincing. If the model could be used to convince people to act badly, then it should not be released. Humans are susceptible to the sentiments of those they see as their peers. If the model was, for better or worse, passing the Turing test, then it should not fall into the wrong hands. This was the concern of the coders at the time.

Ultimately the large model was released, either because the developers decided the model was not as good as originally estimated, or because they didn't care. 

\subsection*{Winograd Schema}

Winograd schemas are named after Terry Winograd. The idea is that there is a sentence presented that has two meanings. A computer finds these sentences challenging to understand, and that makes them interesting for the development of Artificial Intelligence.

An example follows.

\begin{center}
	\textbf{He didn't put the trophy in the suitcase because it was too [big/small]}
\end{center}

We can choose which bracketed term to use, and we must choose only one bracketed term. If we choose `big' then we are referring to the trophy. If we choose `small' then we are referring to the suitcase. Human beings can easily see the pronoun `it' refers to either the suitcase or the trophy. Computers have trouble with these determinations.

The Transformer, and the Scaled Dot-product Attention that it uses, lends itself to discussion of Winograd schema. In the chat bot example, we are less interested in the Winograd example because it doesn't come up often. However, in the case of the Generative Pre-training Transformer 2, and it's exhaustive training, it is interesting to consider the Winograd style example sentences.

There is a Winograd Schema Challenge and something of a formula for constructing your own Winograd schema (Wikapedia contributors, 2020). \cite{wiki:xxx}



\appendix

\chapter{Terminology}




\section{Transformer}

In this section we discuss the `Transformer' model. In their paper, Vaswani et al (2018)\cite{tensor2tensor} describe use of the python project Tensorflow and the Transformer model which is part of it.

The Transformer uses no recurrent elements. It is in a sense a group of attention mechanisms. The Transformer in this case is a single structure that can be used to solve many machine learning problems. It is used for Neural Machine Translation, Sentiment Analysis, and others. It is a single model capable of solving many machine learning questions.

We use the translation models for the chatbot problem by feeding the model the English language on both input and output. 

The Transformer itself can be configured for sentence long output but it is not a pre-trained model. There are pre-trained versions of the transformer, one of which is called BERT. BERT is described in Devlin et al (2018)\cite{DBLP:journals/corr/abs-1810-04805} and the acronym stands for Bidirectional Encoder Representations from Transformers. Unfortunately BERT output is as a 
classifier. Full sentence-length output is not supported.

\section{Generative Pre-training Transformer 2}

Another pre-trained model that uses the Transformer is GPT2. GPT2 stands for `Generative Pre-training Transformer 2.'

GPT2 is a model that takes as input a seed sentence or topic and returns as output text in the same language that is auto-generated. GPT2 also is very capable at summarizing the input or seed statement. We use both of these capabilities in our experiments.

There are two GPT2 models. One is larger than the other. The smaller model has been released and the larger model has not. 

GPT2 is discussed in Radford et al (2019)\cite{radford2019language} and on the blog associated with the parent company, OpenAI.com. (https://openai.com/blog/better-language-models/)

GPT2 comes with its own vocabulary encoding and decoding functionality. This system is closer to
WordPiece and BPE than it is to Glove or Word2Vec.

It is pre-trained on a corpus that is developed from Reddit posts called WebText. WebText is a
40 Gigabyte corpus that takes high powered computers to train.

The version of GPT2 which has been released is similar in size to the largest currently released 
BERT transformer model. GPT2 has been released in Tensorflow format, and more recently in a converted PyTorch format. It is possible to download the trained model and then fine-tune the model for your own task. This kind of fine-tuning is called `transfer learning'. 

GPT2 works so well in certain conditions that it is appropriate to use without fine tuning. This 
type of implementation is called `zero-shot' implementation. We use GPT2 in a zero-shot implementation with the chatbot problem.






\newpage
	
\chapter{Tables}
	
\section{Model Overview}
	
\begin{center}

\begin{table}[h!]
	
\begin{center}


\begin{tabular}{llllll}

	Model Name    & File  & RAM  & RAM  & Pretrained  & Hand  \\
	              &  Size & Train   & Interactive   & Weights & Coded   \\
	\hline
	\hline
	Seq-2-Seq/Tutorial & 230 M     & 1.1 G & 324 M & NO                 & NO        \\
	Transformer/Persona   & 25 M      & 556 M & 360 M & NO         & PARTIAL         \\
	Transformer/Movie   & 550 M      & 6.5G & 1.5G & NO         & PARTIAL         \\
	GPT2 small*   & 523 M     & 5 G   & 1.5 G & YES                & NO        \\
	\hline
\end{tabular}

* a large GPT2 model exists, but it is not small enough to fit on a raspberry pi.

	
\end{center}

\label{fig:modeloverview}
\addcontentsline{lot}{section}{Model Overview}
\end{table}
\end{center}


\iffalse
\section{Question Answering -- babi}

\begin{center}


\begin{tabular}{|l|c|c|c|}
	\hline 
	& {\small{}DMN plus} & {\small{}hand coded}  & {\small{}GPT2 - small}  \tabularnewline
	\hline 
	\hline 
	{\small{}QA1: Single Supporting Fact} & {\small{}100} & {\small{}100} & {\small{}100}  \tabularnewline
	\hline 
	{\small{}QA2: Two Supporting Facts} & {\small{}99.7} & {\small{}xx}  & {\small{}96.0} \tabularnewline
	\hline 
	{\small{}QA3: Three Supporting Facts} & {\small{}98.2} & {\small{}xx} & {\small{}38.18}  \tabularnewline
	\hline 
	{\small{}QA4: Two Argument Relations} & {\small{}100} & {\small{}100}  & {\small{}100} \tabularnewline
	\hline 
	{\small{}QA5: Three Argument Relations} & {\small{}99.5} & {\small{}99.4}  & {\small{}97.8} \tabularnewline
	\hline 
	{\small{}QA6: Yes/No Questions} & {\small{}100} & {\small{}100}  & {\small{}98.4} \tabularnewline
	\hline 
	{\small{}QA7: Counting} & {\small{}97.6} & {\small{}97.8} & {\small{}98.6} \tabularnewline
	\hline 
	{\small{}QA8: Lists/Sets} & {\small{}100} & {\small{}99.4} & {\small{}98.8} \tabularnewline
	\hline 
	{\small{}QA9: Simple Negation} & {\small{}100} & {\small{}98.2}  & {\small{}97.0}  \tabularnewline
	\hline 
	{\small{}QA10: Indefinite Knowledge} & {\small{}100} & {\small{}99.4} & {\small{}96.6} \tabularnewline
	\hline 
	{\small{}QA11: Basic Coreference} & {\small{}100} & {\small{}100} & {\small{}97.6} \tabularnewline
	\hline 
	{\small{}QA12: Conjunction} & {\small{}100} & {\small{}100} & {\small{}99.4} \tabularnewline
	\hline 
	{\small{}QA13: Compound Coreference} & {\small{}100} & {\small{}99.8} & {\small{}95.8} \tabularnewline
	\hline 
	{\small{}QA14: Time Reasoning} & {\small{}99.8} & {\small{}97.2} & {\small{}87.0} \tabularnewline
	\hline 
	{\small{}QA15: Basic Deduction} & {\small{}100} & {\small{}100} & {\small{}64.4} \tabularnewline
	\hline 
	{\small{}QA16: Basic Induction} & {\small{}54.7} & {\small{}48.2} & {\small{}96.39} \tabularnewline
	\hline 
	{\small{}QA17: Positional Reasoning} & {\small{}95.8} & {\small{}59.2} & {\small{}99.0} \tabularnewline
	\hline 
	{\small{}QA18: Size Reasoning} & {\small{}97.9} & {\small{}91.6} & {\small{}100} \tabularnewline
	\hline 
	{\small{}QA19: Path Finding} & {\small{}100} & {\small{}xx} & {\small{}97.3} \tabularnewline
	\hline 
	{\small{}QA20: Agents Motivation} & {\small{}100} & {\small{}100} & {\small{}100} \tabularnewline
	\hline 
\end{tabular}{\tiny \par}
\label{fig:babiresults}
\addcontentsline{lot}{section}{Question Answering -- babi}
	
\end{center}
\fi

\newpage

\chapter{Abbreviations}

\printacronyms[include-classes=abbrev,name=]

%\end{thebibliography}
\newpage

\bibliographystyle{ieeetr}
%\bibliographystyle{alphadin}
\bibliography{proposal_doc}



%\end{multicols}
\end{document}
