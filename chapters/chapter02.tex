
\section{Transformer and Attention}

The Transformer is a mechanism that is based entirely on attention. Strictly speaking this is not the attention explored in the Sequence to Sequence model in Chapter 1, though there are some similarities. It is a model that uses no recurrent components and no convolution components. 

Recurrent components have some negative qualities. Recurrent components are hard to run with batch input data. Also they do not work with very long data strings. 

If a batch of data is to be cycled through a recurrent component, all that data has to go through the component at the first cardinal position before it is cycled through the second or third position. In the first time step everything needs to go through the first RNN module. Then the data can be moved along to the second RNN  module in the second time step.

Also because the RNN is so heavy with Neural Network components, many weights and biases, though they can remember patterns, they loose some information with every pass. This is why there is a practical limit to the length of the input sequences that the typical RNN can use. This is why the length of sentences in network models that use the RNN are short.

Transformers use no RNN components. Their operations can be parallelized so that large batches of data can be processed at once during the same time step. 

Longer sequences can be considered as well, so Transformer input can contain longer english language sentences and even paragraphs.

%\subsection*{Model Overview}
Attention mechanisms are used in a similar way in three places in the model. The first implementation of the Self Attention is discussed below.

It should be noted that input to the Transformer is strings of words from a desired input language. Output is similarly words in a given language. Input words are treated very much the way that they are in Sequence to Sequence models. A word is translated to a number and that number indexes an entry in a word-to-vector table. From that time forward a word is represented by a vector of floating point numbers. In a transformer this word vector can be large. In the original paper Vaswani et al (2017)\cite{Vaswani2017AttentionIA} use a vector size of 512. 

\subsection*{Scaled Dot-Product Attention}

The Transformer has a signature self-attention mechanism. This is possibly one third of the entire Transformer mechanism, but a variety shows up in the other two-thirds. 

The first thing that happens is the input word is converted to three other values. These new vectors are like the input vector but they have a smaller dimensionality. Converting the word vector in this way is accomplished by three simple matrix multiplication operations.

In the diagram below a simple conversion of this type is illustrated. In the diagram we convert a vector with dimension of 1x3 to a dimension of 1x2. In reality we are converting a vector from 1x512 to 1x64. This is a division of 8.



\begin{figure}[H]
	\begin{center}
		
	
	\includegraphics[scale=0.5]{diagram-mat01}
\end{center}
	\caption[Lowering Dimensionality]{Lowering Dimensionality}
	
	%\addcontentsline{lof}{section}{Loss and Accuracy}
\end{figure}


One thing we want to do is to preserve the dimension of our vector. We start with a 512 sized floating point vector and after some processing we want to return to the same size. Before that is done the vector is processed at the smaller size of 64. 

In this self-attention scheme three vectors are actually required. All three vectors are sized 64, and all three are converted by separate matrix multiplication operations. The weights to convert each of the three vectors are different. For this reason the new smaller vectors are all different.

The smaller vectors individually are called q,k, and v. They can also be referred to as larger matrices. The new vector matrices are denoted as Q, K, and V. Q stands for Query. K stands for Key. V stands for Value. 

The Query value is multiplied by the Key values from all vectors in the input. This multiplication is `dot-product' multiplication. When it is done all keys will have low output except those that are closest to the Query. Then the results are passed through a softmax function. When this is complete there will be a single vector that is close to 1 and another group of vectors that are all close to 0.

The vector produced by multiplying the softmax with the V values of every word produces a single word vector that is close to its original value, and many others that are near zero.

This formula from Vaswani et al (2017)\cite{Vaswani2017AttentionIA} 
shows the process.

$$
Attention(Q,K,V)=softmax(\dfrac{QK^T}{\sqrt{d_k}})V
$$

Here the value of $\sqrt{d_k}$ is used to limit the size of the $QK^T$ output and $d_k$ is the dimension 512. Without this the softmax function has to deal with much larger numbers. Smaller numbers for the softmax are preferred. 

The function can actually perform this on large matrices with high dimensionality, in parallel. This parallel matrix operation increases the speed of training.

In the green triangle in the Figure 2.2 we preform the multiplication and selection that was just described.

\begin{figure}[H]
	\begin{center}
		
		
		\includegraphics[scale=0.5]{diagram-mat03-64}
	\end{center}
	\caption[Attention Output]{Attention Output}
	
	%\addcontentsline{lof}{section}{Loss and Accuracy}
\end{figure}




Finally the output we calculated above must be returned somehow to the input dimensionality. This is accomplished by duplicating the procedure described eight times with eight separate weights. When this is done the output of the attention mechanism is concatinated together, returning the output to the proper size.

This multi-headed approach allows different heads to learn different types of relationships, and then when they are summed together the learned relations are recovered and contribute to the output.

\begin{figure}[H]
	\begin{center}
		
	
	\includegraphics[scale=0.5]{diagram-mat02}
\end{center}
	\caption[Matching Input and Output]{Matching Input and Output}
	
	%\addcontentsline{lof}{section}{Loss and Accuracy}
\end{figure}


Later the output is passed through a feed forward network. After that it is combined with the original input again through addition. Then the vectors are normalized. This makes sure that the values are all within reasonable ranges.

This describes the encoder section. There are two other attention segments. Together these two sections combine to form the decoder section.

\subsection*{Transformer - Decoder}
While the encoder takes in the entire input and attends to whatever portion of that input it finds to be important, the decoder is interested in producing one token at a time. 

During inference it produces a token and then it adds to that token, one at a time, until the decoding is finished and something like a sentence is produced. It can attend to any part of the output it has already produced. During training the decoder is exposed to the target sequence under a mask. The mask prohibits the decoder from seeing parts of the target that it should not. This mimics the inference setup and still allows for large input matrices.

\begin{figure}[H]
	\begin{center}
		
		
		\includegraphics[scale=2.0]{diagram-mat04}
	\end{center}
	\caption[Transformer Encoder and Decoder]{Transformer Encoder and Decoder - Vaswani et al(2017)\cite{Vaswani2017AttentionIA}}
	
	%\addcontentsline{lof}{section}{Loss and Accuracy}
\end{figure}

\subsection*{Transformer - General}

The input of the Transformer encoder and decoder employ not only a word-vector table, but also a positional encoding scheme. The model adds to the input vector information that the model can then use to learn the position of words in a sentence. 

Words that are early in the sentence have a certain appearance and words later on appear differently. The Encoder and Decoder use sine and cosine waves to impart this information onto the sentence sequence. 

The Transformer model takes large memory resources, large corpus resources, and large training time resources. Without these components the Transformer is not suitable for many projects.

\subsection*{Pre-Training}
Pre-Training is when the authors of a model train an instance and then make the model available to the public on-line. This is helpful for the average programmer interested in Neural Networks. Training an instance of the transformer model can use up computation resources for days, and require hardware that is costly. Usually the cost of producing a trained model is prohibitively expensive.

After acquiring a trained model, the programmer goes on to adjust the model to their task. Adjusting a pre-trained model to a given task is called `Transfer Learning'. Many tasks lend themselves to Transfer Learning. Conceptually a model can be fine-tuned to any problem and many problems can be addressed with good results after only modest fine-tuning.

There is a pre-trained Transformer model called \ac{BERT}. BERT stands for `Bidirectional Encoder Representations from Transformers'. The BERT files available on-line are mainly for classification tasks. They allow for input that uses a vocabulary size that is very large, but the output is meant to be smaller.

The BERT models use a bidirectional training method. During training the BERT model might be presented with a sentence with a word missing. The training task is to fill in that blank. The model can look at the sentence from any direction in order to find that word.

\section{GPT2}

\ac{GPT2} is a large model. It is based on the Transformer from Vaswani et al (2017)\cite{Vaswani2017AttentionIA} but there are some major changes. The model uses the encoder portion of the Transformer without the decoder. There are some other changes to the output layers. The biggest difference is that it's pre-trained and downloadable.

It still uses Scaled Dot-Product Attention. A model diagram is taken from Radford et al (2018)\cite{radford2018improving}.

\begin{figure}[H]
	\begin{center}
		
		
		\includegraphics[scale=4.0]{diagram-mat05}
	\end{center}
	\caption[GPT2 Encoder]{GPT2 Encoder - Radford et al(2018)\cite{radford2018improving}}
	
	%\addcontentsline{lof}{section}{Loss and Accuracy}
\end{figure}

There are several sizes of pre-trained GPT2 model. They are all rather large. The smallest GPT2 model matches the size of the largest BERT model. The GPT2 models input and output text sequences. In this way they are preferred for our application over the BERT models. 


The GPT2 models are trained on a corpus called WebText. WebText is a 40GB corpus that is taken from the Reddit web site. All the material comes from before 2017 and all the material has a `carma' rating of 3 or better. `Carma' is a rating system used internally on Reddit. 

In their paper Radford et al (2019)\cite{radford2019language} show that their model can generate text from a seed sentence or paragraph. At the time the case was made that the largest GPT2 models should not be released because of their ability to generate text that might fool humans into believing that another person was responsible for the text. Later the larger models were released to the public.

\begin{center}

\begin{tabular}{lrll}
	Size & Parameters & Layers & $d_{model}$ \\
	\hline
	small & 117M       & 12     & 768          \\
	medium & 355M       & 24     & 1024         \\
	large & 774M       & 36     & 1280         \\
	x-large & 1.5B     & 48     & 1600 \\
	xx-large & 8.3B   &  72 &   3072 
\end{tabular}

	
\end{center}
\addcontentsline{lot}{section}{GPT2 Size Overview}

At the time that the first GPT2 model was released the size of the models was mis-stated, but the documentation was not updated immediately. Most values in the table above show sizes that were actually released. The final xx-large model was trained by NVIDIA Applied Deep Learning Research (2019)\cite{2019NVIDIAadlr} and was not released to the public.

The GPT2 models also work in many circumstances in `zero-shot' mode. This is when you use the model pre-trained but without transfer learning. There is no extra training that goes on to make the model suit the task. It is used `as is'.

For the chatbot the GPT2 model with 117 million parameters worked. Some programming was required to make the model output look like chatbot output, but the model itself was not modified.

Later when the larger 774M model was released it was used as a substitution for the 117M model. The test worked, and returned answers that were more well formed than the small model. The larger model does not fit on a Raspberry Pi and so it was not employed on a permanent basis. Using the extra large model was not attempted.

\subsection*{Application Details}
The model is described in Radford et al (2019)\cite{radford2019language} and the accompanying blog post. The model is trained on English without a stated problem. Large neural network models are usually trained with a stated problem in mind. Rather famously this model is used after training to generate English language text. The model takes input from the user, a premise or summary of what is to be generated. The model also takes as input a number called the `temperature.' Then the model generates output. As the `temperature' is set higher the output is more fanciful. 

There is also a tune-able parameter for the output length. Given the ability of the model to invent content, it was determined by the authors that the 'large' model should not be released to the public at first. 

Months later the 'large' model was released. For our chatbot we set the temperature to a low number. We set the length of the output to a sentence-length number of tokens. Then as input we use the output from the speech-to-text translator.

The output is interesting but not useful right away. Heuristics are employed to clean the output and render a short single sentence. This is our final output.

Because the input is meant to be a number of sentences, and because we are using a transformer-based architecture, we have room in the input string to add more information along side the user's question. In this respect the model acts to summarize the input. 

With every input string we include a set of three or four sentences. They include the time, the bot's name, and the bot location and occupation. All of these are invented. What happens is the chatbot summarizes the input and only if the information is relevant then the same information is used as output. Making this possible is the fact that a transformer can accept much longer input strings than a GRU, and generate much longer output strings.