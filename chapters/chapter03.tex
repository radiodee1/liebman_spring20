\section{Approach to the Study}

Several neural network models are used in the project. One is the GRU sequence to sequence model and another the Transformer model for a generative chatbot and finally the Generative Pre-training Transformer 2 model.

We will not try to rewrite the Transformer or GPT2 model ourselves.

In this project we attempt to load as much of our chatbot code onto small board computers as possible. The Raspberry Pi and the Jetson Nano are considered to be small board computers.

We have trained models using the pytorch and tensorflow libraries. These models are responsible for taking in English sentences and producing English output. There is another part of the typical Raspberry Pi setup that includes another neural network component. Speech to text models, which our application requires, rely on large neural network resources. For this purpose we use speech to text resources supplied by Google on the google cloud. To include speech to text libraries locally on the Raspberry Pi would be too costly in computation time and resources like RAM. It would also be complicated to implement technically. It could easily comprise an entire project on its own.

Unfortunately the speech to text resources supplied by Google cost money. To use the service you need to have a billing account with Google.

The speech to text service used on the project and the memory limitations on the Raspberry Pi leads one to ask the question weather the neural network responsible for the chatbot function could not be served from some faster machine located somewhere on the internet. At this time we are not interested in serving these resources. It would entail two calls from the Raspberry Pi for every sentence. This complicates things and also has a time overhead. 

There are several models to test. To test them all would require several servers. In addition they use both Pytorch and tensorflow. Tensorflow has `tensorflow-model-server' for serving models, but Pytorch has no equivalent.

It is important to note that the large Generative Pre-training Transformer 2 model specifically could be served from a remote computer and it would operate faster. 

Currently on the Raspberry Pi decoding a single sentence with the 117M GPT2 takes approximately 13 seconds. Even so, we prefer to install our trained models on the Raspberry Pi directly.

The 117M GPT2 model fits on a Jetson Nano and can produce the response to a question in two or three seconds. The Nano is not much bigger than a Raspberry Pi.

If we were to replace every Raspberry Pi computer in our project with Jetson Nano computers, all of them would respond quickly and with only a natural pause before each reply.

\section{Hardware Installation Overview}



\begin{table}[h]
	
	\begin{center}
		
		
		\begin{tabular}{llllll}
			
			Model Name    & File  & RAM  & RAM    & Hand & Board \\
			&  Size & Train   & Inference    & Trained &   Type \\
			\hline
			\hline
			Seq-2-Seq/Tutorial & 230 M     & 1.1 G & 324 M             & YES  &  Pi 3B \\
			Transformer/Persona   & 25 M      & 556 M & 360 M          & YES  & NONE \\
			Transformer/Movie \dag \dag  & 550 M      & 6.5 G & 500 M  & YES    & Pi 4B  \\
			GPT2 small \dag   & 523 M     & -   & 2.0 G         & NO     &  Pi 4B/Nano \\
			\hline
		\end{tabular}
		
		\bigskip
	\end{center}
		\dag \ a large GPT2 model exists, but it is not small enough to fit on a Raspberry Pi or Jetson Nano.
		
		\dag \dag \ this is the model we refer to as the `larger' Transformer model, though it is smaller than the smallest GPT2 model.
		
	%\end{center}
	
	\label{fig:modeloverview}
	\addcontentsline{lot}{section}{Hardware Overview}
\end{table}


Here is provided a short description for each row in the table.

\begin{itemize}
	\item \textbf{Sequence to Sequence - Tutorial} This model uses the sequence to sequence architecture and the Gated Recurrent Unit component. We hand-coded our own example of this model but it performed poorly. This model is the slightly modified version of the Sequence to Sequence model based on the tutorial from Inkawhich et al \cite{2018Inkawhich}. It actually uses the Movie Dialog corpus. It uses a hidden size of 500 units. This model was used in one Raspberry Pi installation.
	
	\item \textbf{Transformer - Persona} This model uses a Tensorflow Transformer architecture. There was some coding involved to get the model to interface with the text-to-speech and speech-to-text libraries. There was also some coding to load our own corpus data during training. The model parameters describe a rather small model. This model also uses the Persona Dialog corpus. It is not on a Raspberry Pi board. It uses 2 layers, 4  heads, and a hidden size of 128. This model was not permanently installed.
	
	\item \textbf{Transformer - Movie} This model is based on the Transformer model above but uses the Movie Dialog corpus and a parameter set that is larger. In many ways this model is bigger than the model that uses the Transformer and the Persona corpus. It uses 6 layers, 8 heads, and a hidden size of 512. This model was used in one Raspberry Pi installation.
	
	\item \textbf{GPT2 small} This model was downloaded from the internet. It fits on a Raspberry Pi 4B with the 4GB RAM option. It also fits on a Jetson Nano. Some modification was made so that model output was suitable for our purposes. It uses 117 million parameters, with 12 layers and a hidden size of 768. This model was used in one Raspberry Pi installation and one Jetson Nano installation.
\end{itemize}



\section{Setup}

We use linux computers, sometimes with \ac{GPU} hardware for parallel processing. We also use the Python programming language. Code from this project can be run with the 3.x version of Python.

When the project was started we did some programming with Keras using Tensorflow as a backend. Keras was later discarded in favor of Pytorch and Tensorflow. Tensorflow and Pytorch do not work together. Pytorch as a library is still under development at the time of this writing.

Some of the Generative Pre-training Transformer 2 code uses Pytorch. Some of the Transformer and Generative Pre-training Transformer 2 code uses Tensorflow. There is a repository on Github that has the GPT2 trained model using Pytorch instead of Tensorflow.

We use github as a code repository. Code corresponding with this paper can be found at: \href{https://github.com/radiodee1/awesome-chatbot}{https://github.com/radiodee1/awesome-chatbot} . 

As a coding experiment we rewrite the code for the sequence-to-sequence Gated Recurrent Unit model. We have little success with this experiment. We do not rewrite the Generative Pre-training Transformer 2 code from the Tensorflow or Pytorch repository.





\subsection{Graphical Processing Unit vs. Central Processing Unit}

A CPU has a number of cores, a number usually between 2 and 16. A CPU is designed, though, to execute one command at a time. A CPU has limitations when it comes to executing matrix multiplication. Matrix multiplication using a CPU can take a long time.

GPUs, Graphical Processing Units, have the ability to address tasks like matrix multiplication with many more processing units at once. The GPU speeds up parallel processing and has a benefit to neural networking training tasks that the CPU doesn't have.

Unfortunately state of the art neural network models are larger than the capacity of a single GPU. Some models are trained on many GPUs simultaneously. It is not uncommon for a model to train on a computer with eight GPU cards for many days. Training these models is prohibitively expensive for the average programmer. It is possible to rent time on Amazon Web Services or Google cloud with well outfitted computers to do this training, but this can be costly.

This sort of situation is addressed partially by the Transfer Learning scheme. In Transfer Learning someone else trains the model and makes the trained version accessible to the public. Then the average programmer downloads the model and fine tunes it to their task.

In this paper the GRU based Sequence-to-sequence model and the Tensorflow based Transformer model were trained from scratch on a CPU laptop. In the case of the Transformer, several days were required for training. In the GRU example the model trained in less than an hour.

\subsection{Raspberry Pi}

A Raspberry Pi is a small single board computer with an `arm' processor. There are several versions on the market, the most recent of which sports built-in wifi and on-board graphics and sound. The memory for a Raspberry Pi 3B computer is 1Gig of RAM. Recently available, the Raspberry Pi 4B computer can sport 4Gig of RAM.

It has always been the intention that at some time some chatbot of those examined will be seen as superior and will be installed and operated on a Raspberry Pi computer. If more than one model is available then possibly several models could be installed on Pi computers.

For this to work several resources need to be made available. Pytorch needs to be compiled for the Pi. Speech Recognition (\ac{SR}) and Text To Speech (TTS) need to work on the Pi.

For one of the Transformer models to work Tensorflow needs to work on the Pi.

All the files that are trained in the chosen model need to be small enough in terms of their file size to fit on the Pi. Also it must be determined that the memory footprint of the running model is small enough to run on the Pi.

In the github repository files and scripts for the Raspberry Pi are to be found in the \textquoteleft bot\textquoteright{} folder.

Early tests using Google\textquoteright s SR and TTS services show that the Pi can support that type of functionality. 

Google's SR service costs money to operate. Details for setting up Google's SR and TTS functions are beyond the scope of this document. Some info about setting this up can be
found in the README file of this project\textquoteright s github repository.

The pytorch model that is chosen as best will be trained on the desktop computer and then the saved weights and biases will be transferred to the Raspberry Pi platform. The Pi will not need to do any training, only inference. 

\begin{figure}[H]
	\begin{center}
		\includegraphics[scale=0.1]{diagram-pic01}
		
		
	\end{center}
	\caption[Raspberry Pi]{Raspberry Pi 4B - Modified with push button and indicator lights.}
	
	
\end{figure}

\subsection{Jetson Nano}

The Jetson Nano is another small board computer. Though it has no Wifi, it is outfitted with an NVIDIA Graphical Processing Unit. 

While the development computer uses x86\_64 and the Raspberry Pi uses ARMv7, the Nano uses aarch64. It is based on the ARM architecture with 64 bit memory units. ARMv7 uses 32 bit memory units.

It is not much bigger than a Raspberry Pi. It has four Gigabytes of memory and a Ubuntu based operating system. The operating system has Python 3.6 and special NVIDIA GPU libraries pre-installed. It does not have Pytorch out of the box, but Pytorch can be installed.

In normal mode the board has an HDMI connector which provides both video and audio output. For the project the board was outfitted in headless mode. In headless mode the board needs USB connectors to provide audio input and output.

Because it does not have on-board Wifi out of the box, a Wifi board must be installed, or a Wifi USB dongle must be used.

Google's Speech Recognition and Text-To-Speech software works on this platform. The Nano did not need to be employed for any training. It was used solely for inference.



\subsection{Reply Time}

The Google Text-To-Speech, or gTTS software, takes almost no time to execute.

The Speech Recognition, or SR software, takes a few seconds (one or two) to execute, as each audio sample must be sent to a Google server for translation into text.

The actual model is responsible for taking text, provided by the SR software or on the command line, and returning to the calling function a sensible answer in text form. Each model actually does this same operation. Each model is in turn connected to the same SR and gTTS software.

On the development computer most of the models tested take little or no time to execute.

On the Raspberry Pi only the GRU model executes instantaneously. Each of the other models takes some time to execute.

On the Jetson Nano the largest installed model, the 117 M GPT2 model, takes a few seconds to execute. The other models are smaller than the GPT2 model, and so all would take less than two seconds.

To ascertain the time that a model takes to reply one must start with the time the model takes on the respective hardware and then add to that the time for general Speech Recognition server calls.

All models take a few seconds more than normal to reply for the very first time. This leads one to believe that something is wrong with the model. This is normal and is explained by the need for the program to load a large set of matrices for the first reply. This loading overhead is not needed in the subsequent program replies. The matrices only need to load once.

\label{setup-reply-time}

\begin{table}[H]
	
	\begin{center}
		
		
		\begin{tabular}{lllll}
			
			Model Name    & Board  & SR  & Reply    & Total  \\
			&  Type & Time  & Time    & Time  \\
			\hline
			\hline
			Seq-2-Seq/GRU Tutorial & Pi 3B     & 2s & 0s             & 2s   \\
			Transformer/Movie Corpus & Pi 4B      & 2s & 5s  & 7s      \\
			GPT2 small - 117M   & Pi 4B     & 2s   & 13s         & 15s      \\
			GPT2 small - 117M   & Jetson Nano & 2s   & 3s         & 5s      \\
			\hline
		\end{tabular}
		
		\bigskip
	\end{center}
	
	
	\label{fig:time-compare}
	\addcontentsline{lot}{section}{Time Comparison}
\end{table}


\subsection{Tensorflow vs. Pytorch}

Tensorflow is a Google library. Pytorch has it's roots with Facebook. Both run in a Python environment. The learning curve for Tensorflow is steeper than for Pytorch. Pytorch offers the programmer python objects that can be combined to create a neural network. Tensorflow has different pieces that can be combined, but they cannot be examined as easily at run time.

Tensorflow has a placeholder concept for inputting data and getting back results. You set up these placeholders at design time. They are the only way of accessing your data at run time.

Pytorch objects interact with Python more naturally. You can use print statements in your code to show data transferred from one object to another. This is possible at run time.

In favor of Tensorflow, it has a good tool for visualization which can print out all kinds of graphs of your data while your model trains. It is called Tensorboard.

\subsection{Speech and Speech To Text}

Google has python packages that translate text to speech and speech to text. In the case of text to speech the library is called `\ac{gTTS}'. In the case of speech to text the library is called `google-cloud-speech'. 

The gTTS package is simple to use and can be run locally without connection to the internet. The google-cloud-speech package uses a google cloud server to take input from the microphone and return text. For this reason it requires an internet connection and an account with Google that enables Google cloud api use. Google charges the user a small amount for every word that they translate into text. 

Both of these resources, the text-to-speech and speech-to-text, work well on the Raspberry Pi, but configuring speech-to-text for the Pi is not trivial. The user must register a billing account with Google cloud services. In return for this registration the user is able to download a json authentication file. The file must be copied to the Raspberry Pi. 

Furthermore an environment variable must be set that points to the authentication file. The variable is called `GOOGLE\_APPLICATION\_CREDENTIALS'. This environment variable has to be set up before the respective model runs. When the model is launched on startup it may not be launched as a regular user. The model may be launched as, for example, the root user. Somehow the environment variable must be set along with the launching of the neural network model.

The operating system on the Raspberry Pi is based on Debian Linux. In this operating system there is a file which is run immediately after the basic system starts up. This script is called `\textbf{/etc/rc.local}'. It is sufficient to put the environment variable there and follow it with the launching of the model. To ensure that the process goes without a hitch, we attempt to combine the setting of the environment variable with the launching of the program in a single line of code.


\subsection{Corpus Considerations}

We have collected several data sets for the training of a chatbot model. Firstly we have a corpus of movie subtitles. Then we have the corpus described by Mazar{\'{e}} et al \cite{DBLP:journals/corr/abs-1809-01984}. This corpus is designed for training the chatbot task specifically. This is referred to as the Persona corpus.

For the Persona corpus the text is organized into `JSON' objects. There are several different repeated labels. Some of the text is meant to be used in question and answer pairs. There is also some very specific information there that is not organized in this kind of pattern. When we take apart the Persona corpus we find that the sentences labeled with the `history' tag are most suited to our task. We record these values only and discard other labels.

When discussing corpus considerations it is important to mention the WebText corpus, which is culled from Reddit. This corpus is very large. This is the material used for the training of the GPT2 model. We do not use this material our selves, but it is a big part of the success of the GPT2. This corpus is about 40Gig of data.

%Because Reddit is so important to the GPT2 model, we experiment with Reddit downloads. It is possible to download months of reddit posts. We do not use these downloads in these experiments.

\section{ARMv7 Build/Compile}

\subsection{Pytorch `torch' Library 1.1.0 For ARMv7}
We compile the Pytorch library for Raspberry Pi. We use several virtualization techniques to do this compilation. The result of those efforts is a Pytorch python 3.7 library for the Raspberry Pi.

On their web site Milosevic et al \cite{2018Milosevic} compile Pytorch 1.1.0 for the Raspberry Pi. We follow their instructions closely. We are able to build the package for the ARMv7 platform.

The instructions called for constructing a change-root environment where a Fedora Core 30 linux system was set up. Then the ARMv7 system was used in the change-root environment to compile the Pytorch library for the 1.1.0 version.

The production laptop used for development ran Ubuntu linux. For this reason a Virtualbox emulation was set up with Fedora Core 30 on it. Inside that emulator the change-root environment was set up. The library was compiled there successfully. 

There are two problems with the resulting built python package. Firstly there is an error in python when importing the torch library. The error reads `ImportError: No module named \_C'. 

After some research it is clear that the build process for ARMv7 creates some shared object files that are misnamed. A fix is to find the misnamed files and make copies of them with a suitable name. The same outcome could be assured by making symbolic links to the misnamed files with proper names.

There are three files misnamed. They can be found at `\textbf{/usr/lib/python3.7/site-packages/torch/}'. They are named with the same convention. They all have the ending `.cpython-37m-arm7hf-linux-gnu.so'. We want to rename them with the much shorter `.so'. The files are then named `\textbf{\_C.so}', `\textbf{\_dl.so}', and `\textbf{\_thnn.so}'.

This takes care of the `ImportError'. The second problem is that the version of GLIBC in the change-root environment does not match the GLIBC library in the Raspberry Pi Raspbian distribution. This produces the following error: `\textbf{ImportError: /usr/lib/x86\_64-linux-gnu/libstdc++.so.6: version `GLIBCXX\_3.4.26' not found}'.

This is solved by rebuilding the package with Fedora Core 29 instead of 30. 
 
\subsection{Pytorch `torch' Library 1.4.0 For ARMv7}
We recompile the Pytorch library for the Raspberry Pi. We use debian virtualization techniques for the compilation. Because Ubuntu is a Debian derivative it is not necessary to run the process in a Virtualbox container. 

In addition to this, the files created by the compilation are properly named. There is no need to go to the directory `\textbf{/usr/lib/python3.7/site-packages/torch/}' to change anything. 

The time spent compiling the software is approximately 5 hours. Time spent with the Virtualbox container was easily twice that. The time spent on the Raspberry Pi executing a single Generative Pre-training Transformer 2 question and answer remains about 13 seconds, so there was no gain in that respect.

There were several small hurdles to completing the compilation. Firstly the `debootstrap' command needed to be employed at the start. Debian Stretch was used as the host operating system. It was felt that if it was used that the GLIBC compatibility problem would not be faced. This turned out to be the case.

There are some dependencies that need to be installed on the `chroot' environment for Pytorch to compile. One of these is that is important is `libblas3.'

Then Python 3.7 needed to be built on Stretch. The Stretch program repositories use Python 2.7 and 3.5 . The Raspbian operating system on the Raspberry Pi 4B is based on Debian Buster and uses Python 3.7. After compiling Python 3.7 the Git program needed to be compiled from scratch. Git on Stretch has a issue that is fixed upstream, but we want to use Stretch because of the GLIBC issue. Instead of using the upstream fix, we compile Git ourselves.

It is conceivable that the GLIBC issue would not be important if the `chroot' environment used Debian Buster, since that is the basis for the current Raspbian operating system. The Stretch operating system solution works though.

Finally for Pytorch we disable CUDA and distributed computing as neither exists on the Raspberry Pi.

\subsection{Docker Container `tensorflow-model-server' For ARMv7}
The Google machine learning library for python uses a standalone program called `tensorflow-model-server' for serving all tensorflow models in a standard way. The program has not been officially compiled for ARMv7. There exists, though, a docker image that will run on ARMv7.

Docker can be run on the Raspberry Pi in the ARM environment. Below is a terminal excerpt that shows how to do this. These commands are executed on the Pi.

\begin{verbatim}
$ sudo apt-get update
$ sudo apt-get upgrade
$ curl -fsSL test.docker.com -o get-docker.sh 
$ sh get-docker.sh
$ sudo usermod -aG docker $USER
\end{verbatim}

After the last command you need to log out and then log in again to take advantage of the newly installed docker.

The original idea was to find a docker image that used tensorflow-model-server. It might be possible to find and compile a Docker Container for the ARMv7. Then the executable would be removed from the container and used natively on the Raspberry Pi. 

It was found that there existed a version of the Docker Container for the model server that ran on the Raspberry Pi. All that remained was to write a Docker Container script that interacted with the existing ARMv7 container. The author of the original container is Erik Maciejewski \cite{2020Maciejewski}.

`Tensorflow-model-server' is used on the localhost internet address, 127.0.0.1, with a port of 8500. tensorflow-model-server is meant for serving neural network resources on the internet, but with careful planning it works on the Raspberry Pi.

